\documentclass[12pt,b5paper]{standalone}
\usepackage{fontspec}
\usepackage{polyglossia}
\usepackage{circuitikz}
\usepackage{ifthen}   
\usepackage{amsmath}
\usetikzlibrary{calc}
%
\setmainlanguage{english}
\setotherlanguages{arabic}
\newfontfamily\arabicfont[Scale=1.0,Script=Arabic]{Scheherazade}
\newfontfamily\urdufont[Scale=1.0,Script=Arabic]{XB Tabriz}

%transformer outer dimensions
\pgfmathsetmacro{\h}{2.5}
\pgfmathsetmacro{\w}{2}
\pgfmathsetmacro{\t}{0.4}
\pgfmathsetmacro{\g}{0.2}   %gap between coil and edge of window
\pgfmathsetmacro{\nL}{4}       %one less than the number of left hand turns. hence for 4 turns enter 3 here
\pgfmathsetmacro{\nR}{3}      %one less than the number of left hand turns


%window size
\pgfmathsetmacro{\winH}{\h-2*\t}
\pgfmathsetmacro{\winW}{\w-2*\t}
%coil step
\pgfmathsetmacro{\stepHL}{(\h-2*\t-2*\g)/(\nL+0.5)}
\pgfmathsetmacro{\stepHR}{(\h-2*\t-2*\g)/(\nR+0.5)}

\begin{document}
\begin{urdufont}
\begin{tikzpicture}
%grid
%\draw[gray,thick] (-1,-1) grid (3,3);
%\draw[gray,thin,xstep=0.1,ystep=0.1] (-1,-1) grid (3,3);
%transformer
\draw(0,0) rectangle (\w,\h);
\draw(\t,\t) rectangle ++(\winW,\winH);
%----------------------
%left hand coil. top to bottom
\draw(\t/4,{\h-\t-\g}) to [out=0,in=45] (\t,{\h-\t-\g-\stepHL/2}); %top half section
%coil itself
\def\I{-1}
\pgfmathsetmacro{\nLend}{\nL-1}
\pgfmathsetmacro{\Start}{\I+1}
\foreach \y in { 0, ..., \nLend }{
\draw (0,{\h-\t-\g-\stepHL/2-\y*\stepHL}) to [out=-135,in=45] (\t,{\h-\t-\g-\stepHL/2-\y*\stepHL-\stepHL});
} 
%left hand terminals
\draw (\t/4,{\h-\t-\g})--++(-1.25*\t,0) node(T.A1){};
\draw (0,\t+\g)--++(-1*\t,0)node(T.A2){};
%-------------
%right hand coil. top to bottom
\draw(\w-\t/4,{\h-\t-\g}) to [out=180,in=135] (\w-\t,{\h-\t-\g-\stepHR/2}); %top half section
\def\I{-1}
\pgfmathsetmacro{\nRend}{\nR-1}
\pgfmathsetmacro{\Start}{\I+1}
\foreach \y in { \Start, ..., \nRend }{
\draw (\w,{\h-\t-\g-\stepHR/2-\y*\stepHR}) to [out=-45,in=135] (\w-\t,{\h-\t-\g-\stepHR/2-\y*\stepHR-\stepHR});
} 
%right hand terminals
\draw (\w-\t/4,{\h-\t-\g})--++(1.25*\t,0) node(T.B1){};
\draw (\w,\t+\g)--++(1*\t,0)node(T.B2){};
\end{tikzpicture}
\end{urdufont}
\end{document}
%---------------------

