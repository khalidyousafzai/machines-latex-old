\documentclass[b5paper]{standalone}
\usepackage{fontspec}
\usepackage{polyglossia}
\usepackage{circuitikz}
\usepackage{ifthen}   
\usepackage{amsmath}
%\usetikzlibrary{calc}
\usetikzlibrary{calc,decorations.markings,decorations.pathreplacing}
%\usetikzlibrary{calc,decorations.pathreplacing}
%
\setmainlanguage{english}
\setotherlanguages{arabic}
\newfontfamily\arabicfont[Scale=1.0,Script=Arabic]{Scheherazade}
\newfontfamily\urdufont[Scale=1.0,Script=Arabic]{XB Tabriz}



\pgfmathsetmacro{\x}{2}
\pgfmathsetmacro{\y}{3}
\pgfmathsetmacro{\z}{2} 

%\pgfmathsetmacro{\thetaEnd}{180}

\begin{document}
\begin{urdufont}
\begin{tikzpicture}[x={(-0.5cm,-0.5cm)},y={(1cm,0)},z={(0,1cm)}]
%grid
%\draw[gray,thick] (0,0) grid (5,5);
%\draw[gray,thin,xstep=0.1,ystep=0.1] (0,0) grid (5,5);
%AXIS
\draw[gray](0,0,0)--(2.5,0,0);
\draw[gray](0,0,0)--(0,4,0);
\draw[gray](0,0,0)--(0,0,2.5);
%BOX VECTORS
\draw[->](0,0,0)--++(\x/2,0,0) node[gray,above,rotate=45] {\RL{پہلا راستہ}};
\draw(\x/2,0,0)--(\x,0,0)node[left,gray]{$x_0$};
\draw[->](\x,0,0)--(\x,\y/2,0)node[gray,below]{\RL{دوسرا راستہ}};
\draw(\x,\y/2,0)--(\x,\y,0);
\draw[->](\x,\y,0)--(\x,\y,\z/2)node[gray,rotate=90,above]{\RL{تیسرا راستہ}};
\draw(\x,\y,\z/2)--(\x,\y,\z)coordinate(end);
%dashed
\draw[gray,dashed](end)--++(-\x,0,0)--++(0,-\y,0)node[above left]{$\lambda_{20}$};
\draw[gray,dashed](end)++(-\x,0,0)--++(0,0,-\z)node[above right]{$\lambda_{10}$}--++(\x,0,0);
\draw[gray,dashed](end)--++(0,-\y,0)--++(-\x,0,0);
\draw[gray,dashed](end)++(0,-\y,0)--++(0,0,-\z);
%ACTUAL PATH
\draw[thick,->](0,0,0) to [out=-30,in=-150] (\x/2,\y/2,\y/3)node[gray,above,rotate=40]{\RL{اصل راستہ}};
\draw[thick] (\x/2,\y/2,\y/3)  to [out=30,in=180] (\x,\y,\z);
%
\draw node[right] at (0,4,1.5){$W_m(x_0, \lambda_{10},\lambda_{20})$};
\draw[gray,dashed,<-]  (\x,\y,\z)++(0.1,0.1,0) to [out=-30,in=180](0,4,1.5);
%
\end{tikzpicture}
\end{urdufont}
\end{document}
%---------------------

