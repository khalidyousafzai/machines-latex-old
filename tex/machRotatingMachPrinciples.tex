\باب{گھومتے مشین کے بنیادی اصول}
اس باب میں مختلف گھومتے مشین کے بنیادی اصول پر غور کیا جائے گا۔ظاہری طور پر مختلف مشین ایک ہی قسم کے اصولوں پر کام کرتے ہیں جنہیں اس باب میں اکٹھا کیا گیا ہے۔

\حصہ{قانونِ فیراڈے}
فیراڈے کے قانون\فرہنگ{فیراڈے!قانون}\حاشیہب{Faraday's law}\فرہنگ{Faraday's law} کے تحت جب بھی ایک لچھے کا اِرتَباطِ بہاؤ  \عددیء{\lambda} وقت کے ساتھ تبدیل ہو تو اس لچھے میں برقی دباؤ پیدا ہوتا ہے۔ یعنی
\begin{align}
e=-\frac{\partial \lambda}{\partial t}=-N \frac{\partial \phi}{\partial t}
\end{align}
گھومتے مشین میں اِرتَباطِ بہاؤ کی تبدیلی مختلف طریقوں سے لائی جاتی ہے۔ یا تو لچھے کو ساکن مقناطیسی بہاؤ میں گھمایا جاتا ہے، یا پھر ساکن لچھے میں مقناطیس گھمایا جاتا ہے، وغیرہ وغیرہ۔

لچھے مقناطیسی مرکز\فرہنگ{مرکز}\حاشیہب{magnetic core}\فرہنگ{core}  پر لپیٹے جاتے ہیں۔ اس طرح کم سے کم مقناطیسی دباؤ سے زیادہ سے زیادہ مقناطیسی بہاؤ حاصل کیا جاتا ہے اور لچھوں کے مابین مشترکہ مقناطیسی بہاؤ بڑھایا جاتا ہے۔ دیگر یہ کہ مرکز کی شکل تبدیل کر کہ مقناطیسی بہاؤ کو ضرورت کی جگہ پہنچایا جاتا ہے۔

چونکہ ایسے مشین کے مرکز میں مقناطیسی بہاؤ وقت کے ساتھ تبدیل ہوتا ہے لہٰذا مرکز میں بھنور نما برقی رو\فرہنگ{بھنور نما برقی رو}\فرہنگ{برقی رو!بھنور نما}\حاشیہب{eddy currents}\فرہنگ{eddy currents} پیدا ہوتا ہے۔ ان بھنور نما برقی رو کو کم سے کم کرنے کی خاطر، مرکز کو باریک لوہے کی پتری\فرہنگ{پتری}\حاشیہب{laminations}\فرہنگ{laminations} تہہ در تہہ رکھ کر بنایا جاتا ہے ۔ یہ بالکل اسی طرح ہے جیسے ٹرانسفارمروں میں کیا جاتا ہے۔

\حصہ{معاصر مشین}
شکل  میں برقی جنریٹر کا ایک بنیادی شکل دکھایا گیا ہے۔ اس کے مرکز میں ایک مقناطیس ہے جو کہ گھوم سکتا ہے۔ مقناطیس کا مقام اس کے میکانی زاویہ \عددیء{\theta_m} سے بتلائی جاتی ہے۔ اگر مقناطیس کے محور سے رداس کی جانب  ایک لکیر کھینچی جائے، اور اس کو صفر زاویہ تصور کیا جائے، تو \عددیء{\theta_m} اس لکیر سے، گھڑی کی اُلٹی سمت، ناپی جائے گی۔اگرچہ یہ صفر زاویہ طے کرنے والی لکیر کہیں بھی ہو سکتی ہے، اس کتاب میں ہم یہ لکیر مقناطیس کے محور سے،  دائیں ہاتھ یعنی اُفقی سطح، رداس کی سمت میں کھینچے گے۔ اس شکل میں ایسا ہی دکھایا گیا ہے۔

یہاں کچھ باتیں وضاحت طلب ہیں۔ اگر مقناطیس ایک مقررہ رفتار سے یوں گھوم رہا ہو کہ یہ ہر سیکنڈ میں  \عددیء{n}  مکمل چکر لگائے تو ہم کہتے ہیں کہ مقناطیس کے گھومنے کی تعدد \عددیء{n}  ہرٹز\حاشیہب{Hertz} ہے۔اسی بات کو یوں بھی بیان کیا جاتا ہے کہ مقناطیس \عددیء{60n} چکر فی منٹ\فرہنگ{چکر فی منٹ}\حاشیہب{rounds per minute, rpm} کی رفتار سے گھوم رہا ہے۔ آپ جانتے ہیں کہ ایک چکر \عددیء{360\degree} زاویہ یا \عددیء{2 \pi} ریڈیئن\حاشیہب{radians}  پہ مشتمل ہوتا ہے۔ لہٰذا اسی گھومنے کی رفتار کو \عددیء{2\pi n} ریڈیئن فی سیکنڈ بھی کہا جا سکتا ہے۔اس بات کو اب ہم یوں بیان کر سکتے ہیں۔ اگر مقناطیس کے گھومنے کی تعدد \عددیء{f} ہرٹز ہو تو یہ \عددیء{\omega} ریڈیئن فی سیکنڈ کی رفتار سے گھومتا ہے۔ جہاں
\begin{align}
\omega =2\pi f
\end{align}
اس کتاب میں گھومنے کی رفتار عموماً ریڈیئن فی سیکنڈ میں ہی بیان کی جائے گی۔

شکل میں دکھائے گئے مشین میں مقناطیس کے دو قطب ہیں، اس لئے اس کو دو قطب والا مشین کہتے ہیں۔ اس مشین میں ایک لچھا استعمال ہوا ہے جس کی وجہ سے اس کو ایک لچھے کا مشین بھی کہتے ہیں۔ اس کے باہر مقناطیسی مرکز ہے۔ مرکز میں، اندر کی جانب دو  شکاف ہیں، جن میں  \عددیء{N} چکر کا لچھا موجود ہے۔ لچھے کو \عددیء{a} اور \عددیء{-a} سے واضح کیا گیا ہے۔ چونکہ یہ لچھا جنریٹر کے ساکن حصہ پہ پایا جاتا ہے لہٰذا  یہ بھی ساکن رہتا ہے اور اسی وجہ سے اسے  ساکن لچھا\فرہنگ{ساکن لچھا}\حاشیہب{stator coil}\فرہنگ{stator coil} کہتے ہیں۔

 مقناطیس کا مقناطیسی بہاؤ اس کے شمالی قطب\حاشیہب{north pole}  \عددیء{N} سے نکل کر خلائی درز میں سے ہوتا ہوا، باہر گول مرکز میں سے گزر کر اور ایک بار پھر  خلائی درز میں سے ہوتا ہوا مقناطیس کے جنوبی قطب\حاشیہب{south pole}   \عددیء{S} میں داخل ہوتا ہے۔ اس مقناطیسی بہاؤ کو نقطہ دار لکیروں سے دکھایا گیا ہے۔  اگر غور کیا جائے تو یہ  مقناطیسی بہاؤ، سارا کا سارا، ساکن لچھے میں سے بھی گزرتا ہے۔

شکل  میں مقناطیس سیدھے سلاخ کی مانند دکھایا گیا ہے۔ شکل  میں اس مقناطیس کو تقریباً گول دکھایا گیا ہے۔شکل  کی طرح یہاں بھی مقناطیس کے محور کا زاویہ \عددیء{\theta_m} سے ظاہر کیا گیا ہے۔مقناطیس اور باہر مرکز کے درمیان صفر زاویہ، یعنی  \عددیء{\theta=0} ، پر خلائی درز کی لمبائی کم سے کم اور نوے  زاویہ، یعنی \عددیء{\abs{\theta}=90\degree} ، پہ زیادہ سے زیادہ ہے۔ اس کی وضاحت بعد میں کی جائے گی البتہ یہاں اتنا جان لینا ضروری ہے کہ اس طرح خلائی درز میں سائن نما مقناطیسی بہاؤ پیدا کرنا ممکن ہوتا ہے۔ مقناطیسی بہاؤ مقناطیس سے مرکز میں عمودی زاویہ پہ داخل ہوتا ہے۔  اگر مقناطیس اور مرکز کے درمیان خلائی درز میں \عددیء{B} سائن نما ہو، یعنی
\begin{align}
B=B_0 \cos \theta_p
\end{align}
تو خلائی درز میں مقناطیسی بہاؤ \عددیء{B} کی مقدار \عددیء{\theta_p} کے ساتھ تبدیل ہوگی۔یہ کثافتِ مقناطیسی بہاؤ صفر زاویہ،یعنی \عددیء{\theta_p=0}، پہ زیادہ سے زیادہ ہو گی اور نوے زاویہ، یعنی \عددیء{\abs{\theta_p}=90\degree} ، پہ صفر ہو گی۔\عددیء{\theta_p} کو مقناطیس کے شمالی قطب یعنی نکتہ دار اُفقی لکیر سے گھڑی کی اُلٹی سمت ناپا جاتا ہے۔  شکل  میں  مرکز کے باہر نوک دار لکیروں سے اس کثافتِ مقناطیسی بہاؤ کی مقدار اور اس کی سمت دکھائی گئی ہے۔ آدھے خلائی درز میں کثافتِ مقناطیسی بہاؤ  رداس کی سمت میں ہے اور آدھے میں یہ رداس کے اُلٹ سمت میں ہے۔  یہ شکل میں دکھایا گیا ہے۔ اگر ہم خلائی درز میں کثافتِ مقناطیسی بہاؤ \عددیء{B} اور  زاویہ \عددیء{\theta_p} کا گراف بنائیں تو یہ شکل کی طرح ہوگا۔ شکل  میں مقناطیس کسی اور زاویہ پہ دکھایا گیا ہے۔یہاں یہ سمجھ لینا ضروری ہے کہ کثافتِ مقناطیسی بہاؤ کی مقدار  ہر حالت میں مقناطیس کے شمالی قطب پہ زیادہ سے زیادہ ہو گا اور یہاں اس کا رُخ رداس کی  سمت میں ہو گا۔ شکل میں خلائی درز میں کثافتِ مقناطیسی بہاؤ \عددیء{B}، زاویے \عددیء{\theta_p} اور \عددیء{\theta_m} دکھائے گئے ہیں۔ اس شکل کے لئے ہم لکھ سکتے ہیں
\begin{gather}
\begin{aligned}
B&=B_0 \cos \theta_p\\
\theta_p=\theta-\theta_m
\end{aligned}
\end{gather}
لہٰذا
\begin{align}
B=B_0 \cos (\theta-\theta_m)
\end{align}
شکل  میں مقناطیس اور اس سے پیدا سائن نما مقناطیسی دباؤ دکھایا گیا ہے۔ ایسے مقناطیسی دباؤ کو ہم عموما ً ایک سمتیہ سے ظاہر کرتے ہیں جہاں سمتیہ کا طول مقناطیسی دباؤ کے حیطہ کے برابر ہوتا ہے اور اس کی سمت مقناطیس کی شمال کو ظاہر کرتا ہے۔ شکل  میں ایسا ہی دکھایا گیا ہے۔ یہ سمجھ لینا ضروری ہے کہ اس سمتیہ کی سمت سائن نما مقناطیسی دباؤ کے حیطہ کو واضح کرتا ہے۔ 

 شکل  میں مقناطیس کو کسی ایک لمحہ \عددیء{t_1}  زاویہ \عددیء{\theta_m(t_1)} پہ دکھایا گیا ہے۔ یہاں ساکن لچھے کا اِرتَباطِ بہاؤ \عددیء{\lambda_\theta} ہے۔ اگر مقناطیس، گھڑی کے الٹی سمت، ایک مقررہ رفتار \عددیء{\omega_0} سے  گھوم رہا ہو تو ساکن لچھے میں اس لمحہ \عددیء{e(t)} برقی دباؤ پیدا ہوگا جہاں
\begin{align}
e(t)=\frac{\dif \lambda_\theta}{\dif t}
\end{align}
کے برابر ہے۔چونکہ ہمیں برقی دباو کی قیمت نا کہ اس کے \عددیء{\mp} ہونے سے دلچسپی ہے لہٰذا اس مساوات میں منفی کی علامت کو نظر انداز کیا گیا ہے۔

جب مقناطیس آدھا چکر،یعنی \عددیء{\pi} ریڈیئن،  گھومے تو اس کے دونوں  قطب آپس میں جگہیں تبدیل کر لیں گے۔ لچھے میں مقناطیسی بہاؤ کی سمت اُلٹی ہو جائے گی۔ ساکن لچھے میں اِرتَباطِ بہاؤ اب \عددیء{-\lambda_\theta} ہو جائے گا اور اس میں امالی برقی دباؤ \عددیء{-e(t)} ہو جائیں گے۔ اور جب مقناطیس ایک مکمل چکر کاٹے تو مقناطیس  ایک بار پھر اسی جگہ ہوگا جہاں یہ شکل میں دکھایا گیا ہے۔ ساکن لچھے کا اِرتَباطِ بہاؤ ایک بار پھر \عددیء{\lambda_\theta} ہی ہو گا اور اس میں امالی برقی دباؤ بھی ایک بار پھر \عددیء{e(t)} ہی ہوں گے۔ یعنی مقناطیس اگر \عددیء{\theta_m=2\pi} کا زاویہ طے کرے تو امالی برقی دباؤ کے زاویہ میں \عددیء{\theta_e=2\pi} کی تبدیلی آتی ہے۔ لہٰذا دو قطب کی مشین میں میکانی زاویہ \عددیء{\theta_m} اور برقی زاویہ \عددیء{\theta_e} برابر ہوتے ہیں، یعنی
\begin{align*}
\theta_e=\theta_m
\end{align*}
اس مشین میں اگر مقناطیس \عددیء{n} چکر فی سیکنڈ کی رفتار سے گھومے تو لچھے میں امالی برقی دباؤ \عددیء{e(t)} بھی ایک سیکنڈ میں \عددیء{n} مکمل چکر کاٹے گی۔ ہم کہتے ہیں کہ \عددیء{e(t)} کے تعدد\فرہنگ{تعدد} \حاشیہب{frequency}\فرہنگ{frequency}\عددیء{f_e}   کی مقدار  \عددیء{n} ہرٹز\حاشیہب{Hertz} ہے۔ یعنی اس صورت میں \عددیء{f_e=n}  ہرٹز  یا ہم کسی بھی تعدد کے لئے لکھ سکتے ہیں
\begin{align*}
f_e=f_m
\end{align*}

چونکہ اس مشین میں  میکانی زاویہ \عددیء{\theta_m} اور برقی زاویہ \عددیء{\theta_e} وقت کے سات تبدیل ہوتے بھی آپس میں ایک نسبت رکھتے ہیں لہٰذا ایسے مشین کو معاصر مشین\فرہنگ{معاصر}\حاشیہب{synchronous machine}\فرہنگ{synchronous}  کہتے ہیں۔ یہاں یہ نسبت ایک کی ہے۔ 

شکل  میں چار قطب ولا ایک دور کا معاصر جنریٹر دکھایا گیا ہے۔ چھوٹے مشین میں عموما ً مقناطیس ہی استعمال ہوتے ہیں۔ البتہ بڑے مشین میں برقی مقناطیس\فرہنگ{مقناطیس!برقی}\حاشیہب{electromagnet}\فرہنگ{electromagnet} استعمال ہوتے ہیں۔ شکل  میں ایسا ہی دکھایا گیا ہے۔ دو سے زیادہ قطب والے مشین میں کسی ایک شمالی قطب کو حوالہ متن بنایا جاتا ہے۔ شکل میں اس قطب کو \عددیء{\theta_m} پہ دکھایا گیا ہے اور یوں دوسرا شمالی قطب \عددیء{(\theta_m+\pi)} کے زاویہ پہ ہے۔

 جیسا کہ نام سے واضح ہے، اس مشین میں موجود  مقناطیس کے چار قطب  ہیں۔ ہر ایک شمالی قطب کے بعد ایک جنوبی قطب آتا ہے۔ ایک دور کی آلوں میں مقناطیس کے جتنے  قطب کے جوڑے ہوتے ہیں، اس میں  اتنے ہی ساکن لچھے ہوتے ہیں۔ چونکہ شکل  میں دیئے گئے مشین کے چار قطب یعنی دو جوڑے قطب ہیں،  لہٰذا اس مشین کے ساکن حصہ پہ دو ساکن لچھے لپٹے  گئے ہیں۔ ایک لچھے کو \عددیء{a_1} سے واضح کیا گیا ہے اور دوسرے کو \عددیء{a_2} سے۔لچھے \عددیء{a_1} کو  مرکز میں موجود دو شگاف \عددیء{a_1} اور \عددیء{-a_1} میں لپیٹا گیا ہے۔ اسی طرح \عددیء{a_2} لچھے کو دو شگاف \عددیء{a_2} اور \عددیء{-a_2} میں رکھا گیا ہے۔ ان دونوں لچھوں میں یکساں برقی دباؤ پیدا ہوتی ہے۔ ان دونوں لچھوں کو سلسلہ وار\حاشیہب{series connected} جوڑا جاتا ہے۔ اس طرح جنریٹر کی کُل برقی دباؤ ایک لچھے میں پیدا  برقی دباؤ کے دگنا ہوتا ہے۔ایک دور کے آلوں میں  اگر مرکز کو، مقناطیس کے جتنے قطب ہوں اتنے حصوں میں تقسیم کر لیا جائے، تو اس مشین کا ہر ایک ساکن لچھا ایسا ایک حصہ گھیرتا ہے۔ شکل میں چار  قطب  ہیں  لہٰذا اس کا ایک لچھا  نوے میکانی زاویہ کے احاطے کو گھیر رہا ہے۔

اب تک ہم نے گھومتے لچھے اور ساکن لچھے کی بات کی ہے۔یہ دو لچھے دراصل دو بالکل مختلف کارکردگی کے حامل ہوتے ہیں۔اس بات کی  یہاں وضاحت کرتے ہیں۔

جیسا پہلے بھی ذکر ہوا چھوٹی گھومتی آلوں میں مقناطیسی میدان ایک مقناطیس ہی فراہم کرتی ہے جبکہ بڑے آلوں میں برقی مقناطیس یہ میدان فراہم کرتی ہے۔اگرچہ اب تک کی شکلوں میں مقناطیس کو گھومتے حصہ کے طور پر دکھایا گیا ہے مگر حقیقت میں یہ کبھی مشین کا گھومتا حصہ اور کبھی یہ اس کا ساکن حصہ ہوتا ہے۔ میدان فراہم کرنے والا لچھا مشین کے کُل برقی طاقت کے چند فی صد برابر برقی طاقت استعمال کرتا ہے۔اس  میدان فراہم کرنے والے لچھے کو میدانی لچھا\فرہنگ{لچھا!میدانی}\حاشیہب{field coil}\فرہنگ{field coil}  کہتے ہیں۔اس کے برعکس مشین میں موجود دوسری نوعیت کے لچھے کو قوی لچھا\فرہنگ{لچھا!قوی}\حاشیہب{armature coil}\فرہنگ{armature coil}  کہتے ہیں۔برقی جنریٹر سے حاصل برقی طاقت اس قوی لچھے سے ہی حاصل کیا جاتا ہے ۔برقی موٹروں میں میدانی لچھے میں چند فی صد برقی طاقت کے خرچ کے علاوہ بقایا سارا برقی طاقت اسی قوی لچھے کو ہی فراہم کیا جاتا ہے۔

اب اگر ہم، گھومتے اور ساکن حصہ کے درمیان، خلائی درز میں \عددیء{B} کو دیکھیں تو شمالی قطب سے مقناطیسی بہاؤ باہر کی جانب  نکل کر  مرکز میں داخل ہوتا ہے جبکہ جنوبی قطب میں مقناطیسی بہاؤ مرکز سے نکل کر جنوبی قطب میں اندر کی جانب داخل ہوتا ہے۔ یہ شکل  میں دکھایا گیا ہے۔ یوں اگر ہم اس خلائی درز میں ایک گول چکر کاٹیں تو مقناطیسی بہاؤ کی سمت  دو مرتبہ باہر کی جانب اور دو مرتبہ اندر کی جانب ہو گی۔ مزید یہ کہ   آلوں میں کوشش کی جاتی ہے کہ خلائی درز میں \عددیء{B} سائن نما ہو۔ یہ کیسے کیا جاتا ہے، اس کو ہم آگے پڑھے گے۔  لہٰذا اگر یہ تصور کر لیا جائے کہ \عددیء{B} سائن نما ہی ہے تب  خلائی درز میں \عددیء{B} کی مقدار، شکل  کی طرح ہو گی۔ 

 یوں ہم ایک ایسی معاصر مشین جس میں \عددیء{P} قطب مقناطیس پایا جاتا ہو کے لئے لکھ سکتے ہیں
\begin{align}
\theta_e&=\frac{P}{2} \theta_m\\
f_e&=\frac{P}{2} f_m
\end{align}
اس صورت میں میکانی اور برقی تعدد ایک بار پھر آپس میں ایک نسبت رکتے ہیں۔ 
%
\ابتدا{مثال}
پاکستان میں گھروں اور کارخانوں میں \عددیء{\SI{50}{\hertz}} کی برقی طاقت فراہم کی جاتی ہے یعنی ہمارے ہاں \عددیء{f_e=50} ہے۔
\begin{itemize}
\item
اگر یہ برقی طاقت دو قطب کے جنریٹر سے حاصل کی جائے تو یہ جنریٹر  کس رفتار سے گھمایا جائے گا۔
\item
اگر جنریٹر کے بیس قطب ہوں تب یہ جنریٹر کس رفتار سے گھمایا جائے گا۔
\end{itemize}

حل:
\begin{itemize}
\item
مساوات  سے ہم دیکھتے ہیں کہ اگر یہ برقی طاقت دو قطب،\عددیء{P=2}،  والے جنریٹر سے حاصل کی جائے تو اس جنریٹر کو \عددیء{f_m=50} چکر فی سیکنڈ یعنی \عددیء{3000} چکر فی منٹ\حاشیہب{rpm, rounds per minute} گھمانا ہو گا۔
\item
 اگر یہی برقی طاقت بیس قطب، \عددیء{P=20}،  والے جنریٹر سے حاصل کی جائے تو پھر اس جنریٹر کو \عددیء{f_m=5} چکر فی سیکنڈ یعنی \عددیء{300} چکر فی منٹ کی رفتار سے گھمانا ہو گا۔
\end{itemize}
\انتہا{مثال}
%

 اب یہ فیصلہ کس طرح کیا جائے کہ جنریٹر کے قطب کتنے رکھے جائیں۔ درحقیقت پانی سے چلنے والے جنریٹر سست رفتار جبکہ ٹربائن سے چلنے والے جنریٹر تیز رفتار ہوتے ہیں، لہٰذا پانی سے چلنے والے جنریٹر زیادہ قطب رکھتے ہیں جبکہ ٹربائن سے چلنے والے جنریٹر آپ کو دو قطب کے ہی ملیں گے۔

شکل  میں دو قطب والا تین دور کا معاصر مشین دکھایا گیا ہے۔اس میں تین ساکن لچھے ہیں۔ان میں ایک لچھا \عددیء{a} ہے جو مرکز میں  شگاف  \عددیء{a} اور \عددیء{-a} میں رکھا گیا ہے۔ اگر اس شکل میں باقی دو لچھے نہ ہوتے تو یہ بالکل شکل  میں دیا گیا مشین ہی تھا۔البتہ دیئے گئے شکل میں ایک کی بجائے تین ساکن لچھے ہیں۔

 اگر \عددیء{a} لچھا  میں برقی رو یوں ہو کہ شگاف \عددیء{a} میں برقی رو ، کتاب کے صفحہ سے عمودی رُخ میں باہر کی جانب ہو اور  \عددیء{-a} میں برقی رو کا رخ اس کے بالکل الٹ سمت میں ہو تو ہم لچھے کی سمت\فرہنگ{لچھا!سمت} کا تعین دائیں ہاتھ کے ذریعہ یوں کرتے ہیں۔

\begin{itemize}
\item
اگر ہم دائیں ہاتھ کی چار انگلیوں کو دونوں شگافوں میں برقی رو کی جانب لپٹیں تو اسی ہاتھ کا انگوٹھا لچھے کی سمت متعین کرتا ہے۔
\end{itemize}

 شکل  میں لچھا \عددیء{a} کی سمت تیر والی لکیر سے دکھائی گئی ہے۔ اس سمت کو ہم صفر زاویہ تصور کرتے ہیں۔ لہٰذا شکل میں \عددیء{a} لچھا صفر زاویہ پر لپٹا گیا ہے، یعنی \عددیء{\theta_a=0\degree} ہے۔ باقی لچھوں کے زاویہ ، لچھا \عددیء{a} کی سمت سے، گھڑی کی اُلٹی رُخ، ناپے جاتے ہیں۔

شکل  میں لچھا \عددیء{b} کو شگاف \عددیء{b} اور \عددیء{-b} میں رکھا گیا ہے اور لچھا \عددیء{c} کو شگاف \عددیء{c} اور \عددیء{-c} میں رکھا گیا ہے۔ مزید یہ کہ لچھا \عددیء{b} کو \عددیء{120\degree} کے زاویہ پر اور لچھا \عددیء{c} کو \عددیء{240\degree} زاویہ پر رکھا گیا ہے۔ یعنی \عددیء{\theta_b=120\degree} اور \عددیء{\theta_c=240\degree} ہیں۔

 شکل  میں دکھائے گئے لمحہ \عددیء{t_1} پر  اگر لچھے \عددیء{a} کا  اِرتَباطِ بہاؤ \عددیء{\lambda_a(t_1)} ہو تو جب مقناطیس \عددیء{120\degree} کا زاویہ طے کر لے، اس لمحہ \عددیء{t_2} پر  لچھے \عددیء{b} کا اِرتَباطِ بہاؤ \عددیء{\lambda_b(t_2)} ہو گا۔ ہم  دیکھتے ہیں کہ لمحہ \عددیء{t_2} پر مقناطیس اور لچھا \عددیء{b} آپس میں بالکل اسی طرح سے ہیں جیسے \عددیء{t_1} پر مقناطیس اور لچھا \عددیء{a} تھے۔ لہٰذا لمحہ \عددیء{t_2} پر لچھا \عددیء{b} کا اِرتَباطِ بہاؤ بالکل اتنا ہی ہوگا جتنا لمحہ \عددیء{t_1} پر  \عددیء{a} لچھاکا تھا۔ یعنی
\begin{align}
\lambda_b(t_2)=\lambda_a(t_1)
\end{align}
اسی طرح اگر مقناطیس مزید  \عددیء{120\degree} زاویہ طے کرے تو اس لمحہ \عددیء{t_3} پر لچھا \عددیء{c} کا اِرتَباطِ بہاؤ  \عددیء{\lambda_c(t_3)} ہو گا اور مزید یہ کہ یہ \عددیء{\lambda_a(t_1)} کے برابر ہوگا۔یوں
\begin{align}
\lambda_c(t_3)=\lambda_b(t_2)=\lambda_a(t_1)
\end{align}
ہیں۔ان لمحات پر ان  لچھوں میں
\begin{align}
e_a(t_1)&=\frac{\dif \lambda_a(t_1)}{\dif t}\\
e_b(t_2)&=\frac{\dif \lambda_b(t_2)}{\dif t}\\
e_c(t_3)&=\frac{\dif \lambda_c(t_3)}{\dif t}
\end{align}
ہوں گے۔مساوات    کی روشنی میں
\begin{align}
e_a(t_1)=e_b(t_2)=e_c(t_3)
\end{align} 

اگر شکل میں صرف لچھا \عددیء{a} پایا جاتا تو یہ بالکل شکل   کی طرح ہوتا اور اب اگر اس میں مقناطیس کو گھڑی کی اُلٹی سمت ایک مقررہ رفتار \عددیء{\omega_0} سے گھمایا جاتا تو، جیسے پہلے تذکرہ کیا گیا ہے، لچھے \عددیء{a} میں سائن نما برقی دباؤ پیدا ہوتی۔شکل  میں کسی ایک لچھے کو کسی دوسرے لچھے پر کوئی برتری حاصل نہیں۔ لہٰذا اب شکل  میں اگر مقناطیس اسی طرح گھمایا جائے تو اس میں موجود تینوں ساکن لچھوں میں سائن نما برقی دباؤ پیدا ہو گی البتہ مساوات    کے تحت یہ برقی دباؤ آپس میں  \عددیء{120\degree} کے زاویہ پر ہوں گے۔

\حصہ{محرک برقی دباؤ}
قانونِ لورینز\فرہنگ{قانون!لورینز}\حاشیہب{Lorentz law}\فرہنگ{Lorentz law} کے تحت اگر چارج  \عددیء{q} مقناطیسی میدان \سمتیہ{B} میں سمتی رفتار \سمتیہ{v} سے حرکت کر رہا ہو تو اس پر قوت  \سمتیہ{F} اثر کرے گی  جہاں
\begin{align}
\kvec{F}=q (\kvec{v} \times \kvec{B})
\end{align}
کے برابر ہے۔

یہاں سمتی رفتار سے مراد چارج کی سمتی رفتار ہے لہٰذا مقناطیسی میدان کو ساکن تصور کر کے اس میں برقی چارج کی سمتی رفتار \سمتیہ{v} ہو گی۔

اس قوت کی سمت دائیں ہاتھ کے قانون سے معلوم کی جاتی ہے۔اگر یہ چارج شروع کے نقطہ سے آخری نقطہ تک سمتی فاصلہ  \سمتیہ{l} طے کرے تو اس پر \عددیء{W} کام ہوگا جہاں
\begin{align}
W=\kvec{F} \cdot \kvec{l}=q (\kvec{v} \times \kvec{B}) \cdot \kvec{l}
\end{align}
اکائی مثبت چارج کو ایک نقطہ سے دوسرے نقطہ منتقل کرنے کے لئے درکار کام کو ان دو نقطوں کے مابین  برقی دباؤ\فرہنگ{برقی دباؤ}\حاشیہب{potential difference, voltage}\فرہنگ{voltage} کہتے ہیں اور اس کی اکائی وولٹ\فرہنگ{وولٹ}\حاشیہب{volt}\فرہنگ{volt}  \عددیء{\si{\volt}} ہے۔یوں اس مساوات سے ان دو نقطوں کے مابین حاصل برقی دباؤ
\begin{align}
e=\frac{W}{q}= (\kvec{v} \times \kvec{B}) \cdot \kvec{l}
\end{align}
وولٹ ہو گی۔

اس طرح حرکت کی مدد سے حاصل برقی دباؤ کو محرک برقی دباؤ\فرہنگ{برقی دباؤ!محرک}\حاشیہب{electromotive force, emf}\فرہنگ{electromotive force}\فرہنگ{emf}  کہتے ہیں۔ روایتی طور پر کسی بھی طریقہ سے حاصل برقی دباؤ کو محرک برقی دباؤ کہتے ہیں۔ یوں کیمیائی برقی سیل وغیرہ کی برقی دباؤ بھی محرک برقی دباؤ کہلاتی  ہے۔

اس مساوات کو شکل  میں استعمال کرتے ہیں۔ گھومتے حصہ پر ایک چکر کا لچھا نسب ہے۔بائیں جانب خلاء میں لچھے کی برقی تار پر غور کریں۔مساوات  کے تحت اس تار میں موجود مثبت چارج پر صفحہ کی عمودی سمت میں باہر کی جانب قوت اثرانداز ہوگی اور اس میں موجود منفی چارج پر اس کی اُلٹ سمت قوت عمل کرے گی۔اسی طرح مساوات  کے تحت صفحہ سے باہر جانب برقی تار کا سِرا برقی دباؤ  \عددیء{e} کا مثبت سِرا ہوگا اور صفحہ کی اندر جانب برقی تار کا سِرا برقی دباؤ \عددیء{e} کا منفی سِرا ہوگا۔

اگر گھومتے حصہ کی محور پر نلکی محدد قائم کی جائے تو جنوبی مقناطیسی قطب کے سامنے خلاء میں \سمتیہ{B} رداس کی سمت میں ہے جبکہ شمالی مقناطیسی قطب کے سامنے  خلاء میں  \سمتیہ{B} رداس کی اُلٹ سمت میں ہے۔یوں جنوبی قطب کے سامنے شگاف میں برقی تار \سمتیہز{l}{s}  کے لئے ہم لکھ سکتے ہیں
\begin{gather}
\begin{aligned}
\kvec{v}_s&=v \atheta =\omega r \atheta\\
\kvec{B}_s&=B \ar\\
\kvec{l}_s&=l \az
\end{aligned}
\end{gather}
لہٰذا اس جانب لچھے کی ایک تار میں پیدا محرک برقی دباؤ
\begin{gather}
\begin{aligned}
e&=(\kvec{v} \times \kvec{B}) \cdot \kvec{l}\\
&=\omega r B l  (\atheta \times \ar) \cdot \az\\
&=\omega r B l  (-\az) \cdot \az\\
&=-\omega r B l 
\end{aligned}
\end{gather}
ہو گی۔

جنوبی مقناطیسی قطب کے سامنے شگاف میں برقی تار کی لمبائی کی سمت \عددیء{\az} لی گئی ہے۔اس مساوات میں برقی دباؤ کے منفی ہونے کا مطلب ہے کہ برقی تار کا مثبت سِرا \عددیء{-\az} کی سمت میں ہے یعنی اس کا نچلا سِرا مثبت اور اوپر والا سِرا منفی ہے۔یوں اگر اس برقی تار میں برقی رو گزر سکے تو اس کی سمت \عددیء{-\az} یعنی صفحہ کی عمودی سمت میں اندر کی جانب ہوگی جسے شگاف میں دائرہ کے اندر صلیبی نشان سے ظاہر کیا گیا ہے۔ 

اسی طرح شمالی مقناطیسی قطب کے سامنے شگاف میں موجود برقی تار کے لئے ہم لکھ سکتے ہیں\حاشیہط{the angular unit vectors are getting mixed with spherical unit vectors}
\begin{gather}
\begin{aligned}
\kvec{v}_N&=v \atheta=\omega r \atheta\\
\kvec{B}_N&=-B \ar\\
\kvec{l}_N&=l \az
\end{aligned}
\end{gather}
اور یوں 
\begin{gather}
\begin{aligned}
e_N&=(\kvec{v}_N \times \kvec{B}_N)\cdot \kvec{l}_N\\
&=-\omega r B l (\atheta \times \ar) \cdot \az\\
&=-\omega r B l (-\az) \cdot \az\\
&=\omega r B l 
\end{aligned}
\end{gather}

شمالی مقناطیسی قطب کے سامنے شگاف میں برقی تار کی لمبائی کی سمت $\az$ لی گئی ہے۔اس مساوات میں برقی دباؤ کے مثبت ہونے کا مطلب ہے کہ برقی تار کا مثبت سِرا $\az$ کی سمت میں ہے یعنی اس کا اوپر والا سِرا مثبت اور نچلا  سِرا منفی ہے۔یوں اگر اس برقی تار میں برقی رو گزر سکے تو اس کی سمت $\az$ یعنی صفحہ کی عمودی سمت میں باہر کی جانب ہوگی جسے شگاف میں دائرہ کے اندر نقطہ کے نشان سے دکھایا گیا ہے۔ 

یہ دو برقی تار مل کر ایک چکر کا لچھا بناتے ہیں۔ ان دونوں کے نچلے سِرے سلسلہ وار جڑے ہیں جو شکل میں نہیں دکھایا گیا۔یوں اس لچھے کے اوپر نظر آنے والے سروں پر کُل برقی دباؤ \عددیء{e} ان دو برقی تاروں میں پیدا برقی دباؤ  کا مجموعہ ہو گا یعنی
\begin{gather}
\begin{aligned}
e&=2r l B \omega\\
&=A B \omega
\end{aligned}
\end{gather}
یہاں لچھے کا رقبہ  \عددیء{A=2 r l } ہے۔اگر ایک چکر سے اتنی برقی دباؤ حاصل ہوتی ہے تو \عددیء{N} چکر کے لچھے  سے
\begin{gather}
\begin{aligned}
e&=\omega N A B\\
&=2 \pi f N A B\\
&=2 \pi f N \phi
\end{aligned}
\end{gather}
حاصل ہوگا۔

گھومتی آلوں میں خلائی درز میں  \سمتیہ{B} اور \سمتیہ{v}  ہر لمحہ عمودی ہوتے ہیں۔مساوات  سے ظاہر ہے کہ اگر گھومنے کی رفتار اور محوری لمبائی معین ہوں تو پیدا کردہ برقی دباؤ  ہر لمحہ  \عددیء{B} کے براہِ راست متناسب ہوگا۔لہٰذا اگر خلائی درز میں زاویہ کے ساتھ  \عددیء{B} تبدیل ہو تو گھومتے لچھے میں پیدا برقی دباؤ بھی زاویہ کے ساتھ تبدیل ہوگا۔یوں جس شکل کی برقی دباؤ حاصل کرنی ہو اُسی شکل کی کثافتِ مقناطیسی دباؤ خلائی درز میں پیدا کرنی ہوگی۔اگر سائن نما برقی دباؤ پیدا کرنی مقصد ہو تو خلائی درز میں محیط پر سائن نما کثافتِ مقناطیسی بہاؤ ضروری ہے۔

اگلے حصے میں خلائی درز میں ضرورت کے تحت \عددیء{B}  پیدا کرنے کی ترکیب بتلائی جائے گی۔

\حصہ{پھیلے لچھے  اور سائن نما مقناطیسی دباؤ}
ہم نے اب تک جتنے مشین دیکھے ان سب میں لچھے ایک گچھ کی شکل میں تھے۔ مزید یہ کہ ان آلوں میں گھومتے حصے پہ موجود مقناطیس کے اُبھرے قطب\فرہنگ{قطب!ابھرے}\حاشیہب{salient poles}\فرہنگ{pole!salient} تھے۔ درحقیقت آلوں کے عموما ً ہموار قطب\فرہنگ{قطب!ہموار}\حاشیہب{non-salient poles}\فرہنگ{pole!non-salient} ہوتے ہیں اور ان میں پھیلے لچھے\فرہنگ{لچھا!پھیلے}\حاشیہب{distributed winding}\فرہنگ{winding!distributed} پائے جاتے ہیں۔ ایسا کرنے سے ہم ساکن اور گھومتے حصوں کے درمیان خلائی درز میں سائن نما مقناطیسی دباؤ اور سائن نما  کثافتِ مقناطیسی بہاؤ پیدا کر سکتے ہیں۔ 

شکل  میں ایک لچھا گچھ کی شکل کا دکھایا گیا ہے۔اس کے گھومنے والا حصہ گول شکل کا ہے اور اس کا \عددیء{\mu_r \to \infty} ہے۔ساکن حصے کا بھی \عددیء{\mu_r \to \infty} ہے۔ لچھے کا مقناطیسی دباؤ \عددیء{\tau=N i} ہے۔  یہ مقناطیسی دباؤ، مقناطیسی بہاؤ \عددیء{\phi}  کو جنم دیتا ہے جس کو نقطہ دار لکیروں سے ظاہر کیا گیا ہے۔ مقناطیسی بہاؤ کو لچھے کے گرد ایک چکر کاٹتے خلائی درز میں سے دو مرتبہ گزرنا پڑتا ہے۔ لہٰذا
\begin{align}
\tau=N i=2 H l_a
\end{align}
یوں ساکن لچھے کا آدھا مقناطیسی دباؤ ایک خلائی درز اور آدھا دوسرے خلائی درز میں مقناطیسی بہاؤ پیدا کرتا ہے۔ مزید یہ کہ خلائی درز میں کہیں پہ مقناطیسی دباؤ ( اور  مقناطیسی بہاؤ )،  رداس\حاشیہب{radius} کی سمت میں ہیں اور کہیں  پہ خلائی درز میں مقناطیسی دباؤ ( اور مقناطیسی بہاؤ )، رداس کی اُلٹی سمت میں ہیں۔ اگر ہم رداس کی سمت کو مثبت لیں تو   مقناطیسی بہاؤ ( اور مقناطیسی دباؤ ) \عددیء{-\tfrac{\pi}{2} < \theta< \tfrac{\pi}{2}} کے درمیان رداس ہی کی  سمت میں ہیں لہٰذا یہاں  یہ مثبت ہیں جبکہ باقی جگہ  مقناطیسی دباؤ ( اور مقناطیسی بہاؤ ) رداس کی اُلٹ سمت میں ہیں لہٰذا یہاں یہ منفی ہیں۔ ایسا ہی شکل  میں دکھایا گیا ہے۔ اس شکل میں خلائی درز میں مقناطیسی دباؤ کو زاویہ کے ساتھ گراف کیا گیا ہے۔\عددیء{-\tfrac{\pi}{2} < \theta <\tfrac{\pi}{2}} کے درمیان خلائی درز میں مقناطیسی دباؤ \عددیء{\tau_a} لچھے کے مقناطیسی دباؤ \عددیء{\tau} کا آدھا ہے اور اس کی سمت مثبت ہے جبکہ \عددیء{\tfrac{\pi}{2} < \theta <\tfrac{3 \pi}{2}} کی درمیان خلائی درز میں مقناطیسی دباؤ لچھے کے مقناطیسی دباؤ کے آدھا ہے اور اس کی سمت منفی ہے۔ یاد رہے کہ مقناطیسی دباؤ کی سمت\فرہنگ{مقناطیسی دباؤ!سمت} کا تعین رداس کی سمت سے کیا جاتا ہے۔

\جزوحصہ{بدلتی رو والے مشین}
بدلتی رو (اے سی) مشین بناتے وقت یہ کوشش کی جاتی ہے کہ خلائی درز میں مقناطیسی دباؤ سائن نما ہو۔ایسا کرنے کی خاطر لچھوں کو ایک سے زیادہ شگافوں میں تقسیم کیا جاتا ہے۔ اس سے سائن نما مقناطیسی دباؤ کیسے حاصل ہوتی ہے، اس بات کی  یہاں وضاحت کی جائے گی۔

فوریئر تسلسل\فرہنگ{فوریئر تسلسل}\حاشیہب{Fourier series}\فرہنگ{Fourier series} کے تحت ہم کسی بھی تفاعل\حاشیہب{function} \عددیء{f(\theta_p)}  کو یوں لکھ سکتے ہیں۔
\begin{align}
f(\theta_p)=\sum_{n=0}^{\infty} (a_n \cos n \theta_p +b_n \sin n \theta_p)
\end{align}
اگر اس تفاعل کا دوری عرصہ\فرہنگ{دوری عرصہ}\حاشیہب{time period}\فرہنگ{time period} \عددیء{T} ہو تب
\begin{gather}
\begin{aligned}
a_0&=\frac{1}{T} \int_{-T/2}^{T/2} f(\theta_p) \dif \theta_p\\
a_n&=\frac{2}{T} \int_{-T/2}^{T/2} f(\theta_p) \cos n \theta_p \dif \theta_p\\
b_n&=\frac{2}{T} \int_{-T/2}^{T/2} f(\theta_p) \sin n \theta_p \dif \theta_p
\end{aligned}
\end{gather}
کے برابر ہوں گے۔
%
\ابتدا{مثال}\شناخت{مثال_تبادلہ_توانائی_فوریئر_تسلسل}
شکل  میں دیئے گئے مقناطیسی دباؤ کا
\begin{itemize}
\item
فوریئر تسلسل حاصل کریں۔
\item
تیسری موسیقائی جز\فرہنگ{موسیقائی جزو}\حاشیہب{third harmonic component}\فرہنگ{harmonic} اور بنیادی جز\فرہنگ{بنیادی جزو}\حاشیہب{fundamental component}\فرہنگ{fundamental} کی نسبت معلوم کریں۔
\end{itemize}

حل:
\begin{itemize}
\item
مساوات کی مدد سے

\begin{align*}
a_0&=\frac{1}{2\pi} \left[\int_{-\pi}^{-\pi/2} \left(-\frac{Ni}{2} \right) \dif \theta_p+\int_{-\pi/2}^{\pi/2} \left (\frac{Ni}{2}\right) \dif \theta_p +\int_{\pi/2}^{\pi} \left(-\frac{Ni}{2}\right) \dif \theta_p  \right]\\
&=\frac{1}{2\pi} \left[\left(-\frac{Ni}{2} \right)\left(-\frac{\pi}{2}+\pi \right) +\left(\frac{Ni}{2} \right)\left(\frac{\pi}{2}+\frac{\pi}{2} \right)+\left(-\frac{Ni}{2} \right)\left(\pi-\frac{\pi}{2} \right)\right]\\
&=0
\end{align*}
اسی طرح
\begin{align*}
a_n&=\frac{2}{2\pi} \frac{Ni}{2} \left[\int_{-\pi}^{-\pi/2} -\cos n \theta_p \dif \theta_p +\int_{-\pi/2}^{\pi/2} \cos n \theta_p \dif \theta_p+\int_{\pi/2}^{\pi} -\cos n \theta_p \dif \theta_p\right]\\
&=\frac{Ni}{2\pi} \left[-\left. \frac{\sin n \theta_p}{n}\right|_{-\pi}^{-\pi/2} +\left. \frac{\sin n \theta_p}{n}\right|_{-\pi/2}^{\pi/2} -\left. \frac{\sin n \theta_p}{n}\right|_{\pi/2}^{\pi} \right]\\
&=\frac{Ni}{2n\pi} \left[\sin \frac{n\pi}{2}+2\sin \frac{n\pi}{2}+\sin \frac{n\pi}{2} \right]\\
&=\left(\frac{4}{n\pi}\right) \left( \frac{Ni}{2}\right) \sin \frac{n\pi}{2}
\end{align*}
اس مساوات میں \عددیء{n} کی قیمت ایک، دو، تین وغیرہ کے لئے ملتا ہے
\begin{align*}
a_1&=\left(\frac{4}{\pi}\right) \left( \frac{Ni}{2}\right), \quad a_3=-\left(\frac{4}{3\pi}\right) \left( \frac{Ni}{2}\right), \quad a_5=\left(\frac{4}{5\pi}\right) \left( \frac{Ni}{2}\right)\\
a_2&=a_4=a_6=0
\end{align*}
اسی طرح
\begin{align*}
b_n&=\frac{2}{2\pi} \frac{Ni}{2} \left[\int_{-\pi}^{-\pi/2} -\sin n \theta_p \dif \theta_p +\int_{-\pi/2}^{\pi/2} \sin n \theta_p \dif \theta_p+\int_{\pi/2}^{\pi} -\sin n \theta_p \dif \theta_p\right]\\
&=\frac{Ni}{2\pi} \left[\left. \frac{\cos n \theta_p}{n}\right|_{-\pi}^{-\pi/2} -\left. \frac{\cos n \theta_p}{n}\right|_{-\pi/2}^{\pi/2} +\left. \frac{\cos n \theta_p}{n}\right|_{\pi/2}^{\pi} \right]\\
&=0
\end{align*}
\item
ان جوابات سے
\begin{align*}
\abs{\frac{a_3}{a_1}} =\frac{\left(\frac{4}{3\pi}\right) \left( \frac{Ni}{2}\right)}{\left(\frac{4}{\pi}\right) \left( \frac{Ni}{2}\right)}=\frac{1}{3}
\end{align*}
\end{itemize}
حاصل ہوتا ہے۔لہٰذا تیسری موسیقائی جز بنیادی جز کے تیسرے حصے یعنی \عددیء{33.33} فی صد کے برابر ہے۔
\انتہا{مثال}
%
مثال \حوالہ{مثال_تبادلہ_توانائی_فوریئر_تسلسل} میں حاصل کئے گئے \عددیء{a_1,a_2,\cdots} استعمال کرتے ہوئے  ہم خلائی درز میں مقناطیسی دباؤ \عددیء{\tau} کا فوریئر تسلسل یوں لکھ سکتے ہیں۔
\begin{align}\label{مساوات_تبادلہ_توانائی_فوریئر_متقناطیسی_دباؤ_تسلسل}
\tau_a=\frac{4}{\pi}\frac{Ni}{2} \cos \theta_p-\frac{4}{3\pi}\frac{Ni}{2} \cos 3\theta_p+\frac{4}{5\pi}\frac{Ni}{2} \cos 5\theta_p+\cdots
\end{align}
مثال \حوالہ{مثال_تبادلہ_توانائی_فوریئر_تسلسل} سے ظاہر ہے کہ مقناطیسی دباؤ کے موسیقائی اجزاء  کی قیمتیں اتنی کم نہیں کہ انہیں رد کیا جا سکے۔جیسا آپ اس باب میں آگے دیکھیں گے کہ حقیقت میں استعمال ہونے والے  مقناطیسی دباؤ میں موسیقائی اجزاء قابلِ نظر انداز ہوں گے اور ہمیں صرف بنیادی جزو  سے غرض ہو گا۔اسی حقیقت کو مد نظر رکھتے ہوئے ہم  تسلسل کے موسیقائی اجزاء کو نظر انداز کرتے ہوئے اسی مساوات کو یوں لکھتے ہیں۔ 
\begin{align}
\tau_{a}=\frac{4}{\pi}\frac{Ni}{2} \cos \theta_p=\tau_0 \cos \theta_p
\end{align}
جہاں
\begin{align}
\tau_0=\frac{4}{\pi}\frac{Ni}{2} 
\end{align}
کے برابر ہے۔اس مساوات سے ہم دیکھتے ہیں کہ شکل  میں لچھے سے حاصل مقناطیسی دباؤ بالکل اسی طرح ہے جیسے شکل  میں سلاخ نما مقناطیس صفر زاویہ پر رکھے حالت میں دیتا۔ اگر یہاں یہ لچھا کسی ایسے زاویہ پر رکھا گیا ہوتا کہ اس سے حاصل مقناطیسی دباؤ زاویہ \عددیء{\theta_m}  پر زیادہ سے زیادہ ہوتا تو یہ بالکل شکل  میں موجود مقناطیس کی طرح کا ہوتا۔ شکل  ایک ایسی ہی مثال ہے۔ ہم بالکل مساوات   کی طرح اس شکل میں لچھا  \عددیء{a} کے لئے لکھ سکتے ہیں۔
\begin{gather}
\begin{aligned}
\tau_a&=\tau_0 \cos \theta_{p_a}\\
\theta_{p_a}&=\theta-\theta_{m_a}=\theta-0\degree\\
\tau_a&=\tau_0 \cos (\theta-\theta_m)=\tau_0 \cos \theta
\end{aligned}
\end{gather}
اسی طرح لچھا \عددیء{b} اور \عددیء{c} کے  چونکہ \عددیء{\theta_{m_b}=120\degree} اور \عددیء{\theta_{m_c}=240\degree}  لہٰذا ان کے لئے ہم لکھ سکتے ہیں۔
\begin{gather}
\begin{aligned}
\tau_b&=\tau_0 \cos \theta_{p_b}\\
\theta_{p_b}&=\theta-\theta_{m_b}=\theta-120\degree\\
\tau_b&=\tau_0 \cos (\theta-\theta_{m_b})=\tau_0 \cos (\theta-120\degree)
\end{aligned}
\end{gather}
%
\begin{gather}
\begin{aligned}
\tau_c&=\tau_0 \cos \theta_{p_c}\\
\theta_{p_c}&=\theta-\theta_{m_c}=\theta-240\degree\\
\tau_c&=\tau_0 \cos (\theta-\theta_{m_c})=\tau_0 \cos (\theta-240\degree)
\end{aligned}
\end{gather}

اگرچہ ظاہری طور پر خلائی درز میں مقناطیسی دباؤ سائن نما ہرگز نہیں لگتا لیکن مساوات  ہمیں بتلاتی ہے کہ یہ محض آنکھوں کا دھوکہ ہے۔ اس مقناطیسی دباؤ کا بیشتر حصہ سائن نما ہی ہے۔  اب اگر ہم کسی طرح مساوات   میں پہلے رکن کے علاوہ باقی سب رکن کو صفر کر سکیں تو ہم بالکل  سائن نما مقناطیسی دباؤ حاصل کر سکتے ہیں۔

شکل  میں تقسیم شدہ لچھا دکھایا گیا ہے۔ یہاں شکل  میں دکھائے گئے \عددیء{N} چکر کے لچھے کو تین چھوٹے یکساں لچھوں میں تقسیم کیا گیا ہے۔لہٰذا ان میں ہر چھوٹا لچھا \عددیء{\tfrac{N}{3}} چکر کا ہے۔  ایسے چھوٹے لچھوں کو سلسلہ وار جوڑا\فرہنگ{سلسلہ وار}\حاشیہب{series connected} جاتا ہے اور  یوں ان میں یکساں  برقی رو \عددیء{i} گزرے گی۔ ان تین لچھوں کو تین مختلف شگافوں میں رکھا گیا ہے۔پہلے لچھے کو شگاف \عددیء{a_{45}} اور \عددیء{-a_{45}} میں رکھا گیا ہے۔ دوسرے لچھے کو شگاف \عددیء{a_{90}} اور \عددیء{-a_{90}} میں اور تیسرے لچھے کو شگاف \عددیء{a_{135}} اور \عددیء{-a_{135}} میں رکھا گیا ہے۔

شگافوں کے ایک جوڑے کو ایک ہی طرح کے نام دیئے گئے ہیں، البتہ ایک شگاف کو مثبت اور دوسرے کو منفی نام دیا گیا ہے۔یوں شگافوں کا پہلے جوڑا  \عددیء{a_{45}} اور \عددیء{-a_{45}}  ہے۔ شگافوں کے مثبت نام ان کے زاویوں کی نسبت سے رکھے گئے ہیں۔لہٰذا شگاف  \عددیء{a_{45}} درحقیقت \عددیء{45\degree} زاویہ پر ہے، شگاف \عددیء{a_{90}} نبہ درجہ زاویہ پر اور شگاف \عددیء{a_{135}}  ایک سو پینتیس درجہ زاویہ پر ہے۔

چونکہ ہر لچھا \عددیء{\tfrac{N}{3}} چکر کا ہے اور ان سب میں یکساں برقی رو \عددیء{i} ہے،  لہٰذا  شکل میں دیئے گئے پھیلے لچھے سے حاصل مقناطیسی دباؤ کا زاویہ کے ساتھ گراف شکل  کے نچلے گراف کی طرح ہو گا۔اس شکل میں سب سے اُوپر لچھا  \عددیء{a_{45}}  کے مقناطیسی دباؤ کا گراف ہے۔ یہ بالکل  میں دیئے گراف کی طرح ہے البتہ یہ صفر زاویہ سے \عددیء{-45\degree} ہٹ کر ہے۔اُوپر سے دوسرا گراف لچھا \عددیء{a_{90}} کا ہے جو ہو بہو شکل  کی طرح ہے جبکہ اس سے نیچے لچھا \عددیء{a_{135}} کا گراف ہے جو صفر زاویہ سے \عددیء{+45\degree} ہٹ کر ہے۔ان تینوں گرافوں میں طول \عددیء{\tfrac{Ni}{6}} ہے۔

ان تینوں گرافوں سے کُل مقناطیسی دباؤ کا گراف یوں حاصل ہوتا ہے۔اس شکل میں عمودی نقطہ دار لکیریں لگائی گئی ہیں۔ بائیں جانب پہلی لکیر کی بائیں طرف علاقے کو الف کہا گیا ہے۔اس علاقے میں پہلے تینوں گرافوں کی مقدار \عددیء{-\tfrac{Ni}{6}} ہے لہٰذا ان کا مجموعہ \عددیء{-\tfrac{Ni}{2}} ہو گا۔یہی سب سے نچلے کُل مقناطیسی دباؤ کی گراف میں دکھایا گیا ہے۔ اسی طرح علاقہ ب میں پہلے گراف کی مقدار \عددیء{+\tfrac{Ni}{6}} ، دوسری گراف کی \عددیء{-\tfrac{Ni}{6}} اور تیسری کی بھی \عددیء{-\tfrac{Ni}{6}} ہے۔ ان کا مجموعہ \عددیء{-\tfrac{Ni}{6}} بنتا ہے جو کُل مقناطیسی دباؤ ہے۔علاقہ ج میں
 \عددیء{+\tfrac{Ni}{6}}، \عددیء{+\tfrac{Ni}{6}} اور \عددیء{-\tfrac{Ni}{6}} مقداریں ہیں جن کا مجموعہ \عددیء{+\tfrac{Ni}{6}} ہی کُل مقناطیسی دباؤ ہے جو سب سے نچلے گراف میں دکھایا گیا ہے۔ اسی طرح آپ پورا گراف بنا سکتے ہیں۔

شکل  کے نچلے گراف کو شکل  میں دوبارہ دکھایا گیا ہے۔

شکل  کا اگر شکل  کے ساتھ تقابل کیا جائے تو محض دیکھنے سے بھی یہ ظاہر ہے کہ شکل   زیادہ سائن نما موج کے نوعیت کا ہے۔ ہمیں فوریئر تسلسل حل کرنے سے بھی یہی نتیجہ ملتا ہے۔ہم دیکھ سکتے ہیں کہ  شگافوں کی جگہ اور ان میں لچھوں کے چکر کو یوں رکھا جا سکتا ہے کہ ان سے پیدا کردہ مقناطیسی دباؤ سائن نما کے زیادہ سے زیادہ قریب ہو۔

چونکہ پھیلے لچھے کے مختلف حصے ایک ہی زاویہ پہ مقناطیسی دباؤ نہیں بناتے لہٰذا ان سے حاصل کُل مقناطیسی دباؤ کا حیطہ ایک گچھ لچھے  کے حیطہ سے قدرِ کم ہوتا ہے۔اس اثر کو مساوات میں جزو \عددیء{k_w} کے ذریعہ یوں ظاہر کیا جاتا ہے۔
\begin{gather}
\begin{aligned}
\tau_0&=k_w \frac{4}{\pi}\frac{N i}{2}\\
\tau_{a}&=k_w \frac{4}{\pi}\frac{N i}{2} \cos \theta=\tau_0 \cos \theta
\end{aligned}
\end{gather}
اس مساوات میں \عددیء{k_w} کو جزو  پھیلاؤ\فرہنگ{جزو!پھیلاو}\حاشیہب{winding factor}\فرہنگ{winding factor}  کہتے ہیں۔ یہ  اکائی سے قدرِ کم ہوتا ہے یعنی
\begin{align}
0<k_w<1
\end{align}
%
\ابتدا{مثال}
شکل  میں دیئے گئے پھیلے لچھے کے لئے \عددیء{k_w} معلوم کریں۔

حل: شکل  سے رجوع کریں۔ یہ تین چھوٹے لچھے برابر مقناطیسی دباؤ \عددیء{\tau_n=\tfrac{4}{\pi}\tfrac{ni}{2}} پیدا کرتے ہیں، البتہ ان کی سمتیں مختلف ہیں۔یہاں چونکہ ایک لچھا  \عددیء{\tfrac{N}{3}} چکر کا ہے لہٰذا \عددیء{n=\tfrac{N}{3}} ہے۔ ہم ان سمتیوں کو جمع کر کے ان کا مجموعی مقناطیسی دباؤ \عددیء{\tau} معلوم کرتے ہیں۔کل دباؤ \عددیء{2.4142 \tau_n} نکلتا ہے ۔ یعنی
\begin{align*}
\tau=2.4142 \frac{4}{\pi}\frac{ni}{2}=\frac{2.4142}{3} \frac{4}{\pi}\frac{N i}{2}=0.8047 \frac{4}{\pi}\frac{N i}{2}
\end{align*}
لہٰذا \عددیء{k_w=0.8047} کے برابر ہے۔
\انتہا{مثال}
%
\ابتدا{مثال}
ایک تین دور \عددیء{50} ہرٹز پر چلنے والا ستارا نما جڑے جنریٹر کو  \عددیء{3000} چکر فی منٹ کی رفتار سے چلایا جا رہا ہے۔تیس چکر کے میدانی لچھے  کا جزو پھیلاو \عددیء{k_{w,m}=0.9} جبکہ پندرہ چکر قوی لچھے کا جزو پھیلاو \عددیء{k_{w,q}=0.833} ہیں۔مشین کا رداس \عددیء{0.7495} میٹر اور اس  کی لمبائی \عددیء{l=2.828} میٹر ہیں۔خلائی درز \عددیء{l_k=0.04} میٹر ہے۔اگر اس کے میدانی لچھے میں \عددیء{1000}  ایمپیئر برقی رو ہے تو معلوم کریں
\begin{itemize}
\item
میدانی مقناطیسی دباؤ کی زیادہ سے زیادہ مقدار۔
\item
خلائی درز میں کثافتِ مقناطیسی بہاؤ۔
\item
ایک قطب پر مقناطیسی بہاؤ۔
\item
محرک تار پر برقی دباؤ۔
\end{itemize}

حل:
\begin{itemize}
\item
\begin{align*}
\tau_0&=k_{w,m} \frac{4}{\pi}\frac{N_m i_m}{2}=0.9 \times \frac{4}{\pi} \times \frac{30 \times 1000}{2}=\SI{17186}{\ampere \cdot turns \per \meter}
\end{align*}
\item
\begin{align*}
\tau_0&=k_{w,m} \frac{4}{\pi}\frac{N_m i_m}{2}=0.9 \times \frac{4}{\pi} \times \frac{30 \times 1000}{2}=\SI{17186}{\ampere \cdot turns \per \meter}
\end{align*}
\item
\begin{align*}
B_0&=\mu_0 H_0=\mu_0 \frac{\tau_0}{l_k}=4 \pi 10^{-7} \times \frac{17186}{0.04}=\SI{0.54}{\tesla}
\end{align*}
\item
\begin{align*}
\phi_0&=2 B_0 l r =2 \times 0.54 \times 2.828 \times 0.7495=\SI{2.28915}{\weber}
\end{align*}
\item
\begin{align*}
E_{rms}&=4.44 f k_{w,q} N_q \phi_0\\
&=4.44 \times 50 \times 0.833 \times 15 \times 2.28915\\
&=\SI{6349.85}{\volt} 
\end{align*}
\end{itemize}
لہٰذا ستارا جڑی جنریٹر کی تار کی برقی دباؤ
\begin{align*}
\sqrt{3} \times 6349.85 \approx \SI{11000}{\volt}
\end{align*}
ہو گی۔
\انتہا{مثال}
%
جیسا پہلے ذکر ہوا ہم چاہتے ہیں کہ سائن نما مقناطیسی دباؤ حاصل کر سکیں۔ چھوٹے لچھوں کے چکر اور شگافوں کی جگہ یوں چنے جاتے ہیں کہ یہ بنیادی مقصد پورا ہو۔ شکل  میں ہم دیکھتے ہیں کہ صفر زاویہ کی دونوں جانب مقناطیسی دباؤ کی موج یکساں طور پر گھٹتی یا بڑھتی ہے۔ یعنی جمع اور منفی  پینتالیس زاویہ پر مقناطیسی دباؤ  \عددیء{\tfrac{Ni}{3}}  گھٹ جاتی ہے۔ اسی طرح جمع اور منفی نوے زاویہ پر یہ یکساں طور پر مزید گھٹتی ہے، وغیرہ وغیرہ۔ یہ ایک بنیادی اصول ہے جس کا خیال رکھنا ضروری ہے۔

چھوٹے لچھوں کے چکر اور شگافوں کی جگہوں کا فیصلہ فوریئر تسلسل کی مدد سے کیا جاتا ہے۔فوریئر تسلسل میں موسیقائی جُز کم سے کم اور اس میں بنیادی جُز زیادہ سے زیادہ رکھے جاتے ہیں۔

ساکن لچھوں کی طرح حرکت کرتے لچھوں کو بھی ایک سے زیادہ چھوٹے لچھوں میں تقسیم کیا جاتا ہے تا کہ سائن نما مقناطیسی دباؤ حاصل ہو۔

\حصہ{مقناطیسی دباؤ کی گھومتی موجیں}
گھومتے آلوں میں لچھوں کو برقی دباؤ دیا جاتا ہے جس سے اس کا گھومنے والا حصہ حرکت میں آتا ہے۔ یہاں ہم اس بات کا مطالعہ کرتے ہیں کہ یہ گھومنے کی حرکت کیسے پیدا ہوتی ہے۔

\جزوحصہ{ایک دور کی لپٹی مشین}
مساوات  میں ایک لچھے کی مقناطیسی دباؤ یوں دی گئی ہے۔
\begin{align}
\tau_a=k_w \frac{4}{\pi}\frac{Ni}{2} \cos \theta
\end{align}
 اگر اس لچھے میں مقناطیسی بہاؤ بھی سائن نما ہو یعنی
\begin{align}
i_a=I_0 \cos \omega t
\end{align}
تو 
\begin{align}
\tau_a=k_w \frac{4}{\pi} \frac{N I_0}{2} \cos \theta \cos \omega t=\tau_0 \cos \theta \cos \omega t
\end{align}
ہو گا جہاں
\begin{align}
\tau_0=k_w \frac{4}{\pi} \frac{N I_0}{2}
\end{align}
کے برابر ہے۔مساوات  کہتا ہے کہ یہ مقناطیسی دباؤ زاویہ \عددیء{\theta} اور لمحہ \عددیء{t} کے ساتھ تبدیل ہوتا ہے۔ اس مساوات کو ہم مندرجہ ذیل قلیہ سے دو ٹکڑوں میں توڑ سکتے ہیں۔
\begin{align*}
\cos \alpha \cos \beta =\frac{\cos (\alpha +\beta) +\cos (\alpha -\beta)}{2}
\end{align*}
لہٰذا
\begin{align}
\tau_a=\tau_0 \left [\frac{\cos (\theta +\omega t) +\cos (\theta -\omega t)}{2}\right]=\tau_a^{-}+\tau_a^{+}
\end{align}
لکھا جا سکتا ہے۔یوں
\begin{align}
\tau_a^{-}&=\frac{\tau_0}{2} \cos (\theta +\omega t)\\
\tau_a^{+}&=\frac{\tau_0}{2} \cos (\theta -\omega t)
\end{align}
ہیں۔اس مساوات سے یہ بات سامنے آتی ہے کہ درحقیقت یہ مقناطیسی دباؤ دو اُلٹ سمتوں میں گھومنے والے مقناطیسی دباؤ کی موجیں ہیں۔ اس کا پہلا جزو \عددیء{\tau_a^-} زاوایہ \عددیء{\theta} گھٹنے کی جانب گھومتا ہے یعنی گھڑی کی سمت میں اور اس کا دوسرا جزو \عددیء{\tau_a^+} گھڑی کی اُلٹی سمت گھومتا ہے یعنی یہ زاویہ بڑھنے کی جانب گھومتا ہے۔

ایک دور کی لپٹی آلوں میں یہ کوشش کی جاتی ہے کہ ان دو گھومتے مقناطیسی دباؤ میں سے ایک کو بالکل ختم یا کم سے کم کیا جائے۔ اس طرح کرنے سے ایک ہے سمت میں کُل مقناطیسی دباؤ گھومتا ملتا ہے جو بالکل اسی طرح کا ہوتا ہے جیسے ایک مقناطیس گھمایا جا رہا ہو۔ تین دور کے آلوں میں یہ کرنا نہایت آسان ہوتا ہے لہٰذا انہیں پہلے سمجھ لینا زیادہ بہتر ہو گا۔

\جزوحصہ{تین دور کی لپٹی مشین کا تحلیلی تجزیہ}
شکل  میں تین دور کی لپٹی مشین دکھائی گئی ہے۔مساوات ،  اور  میں ایسے تین لچھوں کی فوریئر تسلسل کی بنیادی جُز دیئے گئے ہیں جو کے یہ ہیں۔
\begin{align}
\tau_a&=k_w \frac{4}{\pi}\frac{N_a i_a}{2}\cos \theta\\
\tau_b&=k_w \frac{4}{\pi}\frac{N_b i_b}{2}\cos (\theta-120\degree)\\
\tau_c&=k_w \frac{4}{\pi}\frac{N_c i_c}{2}\cos (\theta+120\degree)
\end{align}
اگر ان تین لچھوں میں تین دوری برقی رو ہو یعنی
\begin{align}
i_a&=I_0 \cos (\omega t+\alpha)\\
i_b&=I_0 \cos (\omega t +\alpha -120\degree)\\
i_c&=I_0 \cos (\omega t +\alpha +120\degree)
\end{align}
تو بالکل مساوات کی طرح ہم مساوات  کی مدد سے مساوات کو یوں لکھ سکتے ہیں۔
\begin{align}
\tau_a&=k_w \frac{4}{\pi}\frac{N_a I_0}{2} \cos \theta \cos (\omega t +\alpha)\\
\tau_b&=k_w \frac{4}{\pi}\frac{N_b I_0}{2} \cos (\theta -120\degree)\cos (\omega t +\alpha -120\degree)\\
\tau_c&=k_w \frac{4}{\pi}\frac{N_c I_0}{2} \cos (\theta +120\degree)\cos (\omega t +\alpha +120\degree)
\end{align}
اگر
\begin{align*}
N_a=N_b=N_c=N
\end{align*}
ہو تو انہیں
\begin{align}\label{مساوات_تبادلہ_تین_گھومتے_دباو}
\tau_a&=\frac{\tau_0}{2} \left[\cos (\theta +\omega t +\alpha) +\cos (\theta -\omega t -\alpha) \right]\\
\tau_b&=\frac{\tau_0}{2} \left[\cos (\theta +\omega t +\alpha-240\degree) +\cos (\theta -\omega t -\alpha) \right]\\
\tau_c&=\frac{\tau_0}{2} \left[\cos (\theta +\omega t +\alpha+240\degree) +\cos (\theta -\omega t -\alpha) \right]
\end{align}
لکھ سکتے ہیں جہاں
\begin{align}
\tau_0=k_w \frac{4}{\pi}\frac{N I_0}{2}
\end{align}
ہے۔کل مقناطیسی دباؤ \عددیء{\tau} ان سب کا مجموعہ ہو گا۔ انہیں جمع کرنے سے پہلے ہم ثابت کرتے ہیں کہ
\begin{align*}
\cos \gamma +\cos (\gamma -240\degree)+\cos (\gamma+240\degree)=0
\end{align*}
کے برابر ہے۔ہمیں معلوم ہے کہ 
\begin{align*}
\cos  (\alpha +\beta)&=\cos \alpha \cos \beta-\sin \alpha \sin \beta\\
\cos  (\alpha -\beta)&=\cos \alpha \cos \beta+\sin \alpha \sin \beta
\end{align*}
اگر ہم \عددیء{\alpha = \gamma} اور \عددیء{\beta=240\degree} لیں تو
\begin{align*}
\cos (\gamma+240\degree)&=\cos \gamma \cos 240\degree-\sin \gamma \sin 240\degree\\
\cos (\gamma-240\degree)&=\cos \gamma \cos 240\degree+\sin \gamma \sin 240\degree
\end{align*}
چونکہ \عددیء{\cos 240\degree=-\tfrac{1}{2}} اور \عددیء{\sin 240\degree=-\tfrac{\sqrt{3}}{2}} لہٰذا
\begin{align*}
\cos (\gamma+240\degree)&=-\frac{1}{2}\cos \gamma +\frac{\sqrt{3}}{2}\sin \gamma\\
\cos (\gamma-240\degree)&=-\frac{1}{2}\cos \gamma -\frac{\sqrt{3}}{2}\sin \gamma
\end{align*}
اب اس مساوات کو اگر ہم  \عددیء{\cos \gamma} کے ساتھ جمع کریں تو جواب صفر ملتا ہے، یعنی
\begin{align*}
\cos \gamma +\cos (\gamma+240\degree)+\cos (\gamma-240\degree)=0
\end{align*}
\عددیء{\gamma=\theta+\omega t +\alpha} کے لئے اس مساوات کو یوں لکھ سکتے ہیں۔
\begin{align}\label{مساوات_تبادلہ_تین_سمتیات_جمع_صفر}
\cos (\theta +\omega t +\alpha) +\cos (\theta +\omega t +\alpha+240\degree)+\cos (\theta+\omega t +\alpha-240\degree)=0
\end{align}
اب ہم  اگر مساوات \حوالہ{مساوات_تبادلہ_تین_گھومتے_دباو}  میں دئے  \عددیء{\tau_a} ، \عددیء{\tau_b} اور \عددیء{\tau_c}  کو جمع کریں اور ان میں مساوات \حوالہ{مساوات_تبادلہ_تین_سمتیات_جمع_صفر}  کا استعمال کریں تو ملتا ہے
\begin{align}\label{مساوات_تبادلہ_گھومتا_موج}
\tau^+=\tau_a+\tau_b+\tau_c=\frac{3 \tau_0}{2} \cos (\theta -\omega t -\alpha)
\end{align}
مساوات \حوالہ{مساوات_تبادلہ_گھومتا_موج} کہتا ہے کہ کُل مقناطیسی دباؤ کا حیطہ کسی ایک لچھے کے مقناطیسی دباؤ کے حیطہ کے \عددیء{\tfrac{3}{2}} گنا ہے۔مزید یہ کہ یہ مقناطیسی دباؤ کی موج گھڑی کی اُلٹی سمت گھوم رہی ہے۔ لہٰذا تین لچھوں کو \عددیء{120\degree}  زاویہ پر رکھنے اور انہیں تین دور کی برقی رو، جو آپس میں \عددیء{120\degree} پر ہوں،  سے  ہیجان کرنے سے ایک ہی گھومتی مقناطیسی دباؤ کی موج وجود میں آتی ہے۔ یہاں اس بات کا ذکر کرنا ضروری ہے کہ اگر کوئی دو برقی رو آپس میں تبدیل کئے جائیں تو مقناطیسی موج کے گھومنے کی سمت تبدیل ہو جاتی ہے۔  یہ مثال میں واضح کیا گیا ہے۔

اب ہم دیکھتے ہیں کہ مساوات \حوالہ{مساوات_تبادلہ_گھومتا_موج} ایک گھومتے موج کو ظاہر کرتی ہے۔یہ کرنے کے لئے ہمیں اس موج کی چوٹی کو دیکھنا ہوگا۔ہم اپنی آسانی کے لئے \عددیء{\alpha}  کو صفر لیتے ہیں۔ اس مثال میں ہم برقی رو کی تعدد  \عددیء{\SI{50}{\hertz}} لیتے ہیں۔ اس موج کی چوٹی درحقیقت \عددیء{\cos (\theta -\omega t)} کی چوٹی ہی ہے لہٰذا ہم اسی کی چوٹی کو مدنظر رکھتے ہیں۔ ہمیں معلوم ہے کہ \عددیء{\cos \alpha} کی زیادہ سے زیادہ مقدار ایک کے برابر ہے یعنی اس کی چوٹی ایک کے برابر ہے اور یہ وہاں ہوتی ہے جہاں \عددیء{\alpha} صفر کے برابر ہو یعنی \عددیء{\cos 0 =1}۔ لہٰذا \عددیء{\cos \alpha} کی چوٹی اسی جگہ ہوگی جہاں \عددیء{\alpha} صفر کے برابر ہوگا۔اسی طرح \عددیء{\cos (\theta -\omega t)} کی چوٹی وہیں ہو گی جہاں \عددیء{(\theta - \omega t)} صفر کے برابر ہو یعنی \عددیء{(\theta-\omega t)=0} پر۔

اب ابتدائی لمحہ  یعنی \عددیء{t=0} پر \عددیء{\cos (\theta -\omega t)} کی چوٹی \عددیء{(\theta-\omega t)=0} پر ہو گی۔ اس کو حل کرتے ہیں۔
\begin{align*}
\theta-\omega t =0\\
\theta -\omega \times 0=0\\
\theta =0
\end{align*}
ہم دیکھتے ہیں کہ موج کی چوٹی صفر برقی زاویہ پر ہے۔یہ شکل  میں دکھایا گیا ہے۔ہم اس چوٹی کو کچھ وقفے کے بعد دوبارہ دیکتے ہیں مثلاً \عددیء{t=0.001} سیکنڈ کے بعد۔
\begin{align*}
&\theta-\omega t =0\\
&\theta -\omega \times 0.001=0\\
&\theta =0.001 \omega =0.001 \times 2 \times \pi \times 50=\SI{0.3142}{\radian} 
\end{align*}
اب یہ چوٹی \عددیء{0.3142} برقی ریڈیئن یعنی \عددیء{18\degree} کے برقی زاویہ پر ہے۔یہ بھی شکل میں دکھایا گیا ہے۔یہ بات واضح ہے کہ مقناطیسی دباؤ کی موج گھڑی کی اُلٹی سمت یعنی زاویہ بڑھنے کی سمت میں گھوم گئی ہے۔ اسی طرح \عددیء{t=0.002}  پر یہ چوٹی \عددیء{36\degree} برقی زاویہ پر نظر آئے گی۔اس چوٹی کا زاویہ کسی بھی لمحہ \عددیء{t'} پر بالکل اسی طرح معلوم کیا جا سکتا ہے۔
\begin{align*}
&\theta-\omega t' =0\\
&\theta =\omega t'
\end{align*}
اس مساوات سے یہ واضح ہے کہ چوٹی کا مقام متعین کرنے والا زاویہ بتدریج بڑھتا رہتا ہے۔اس مساوات سے ہم ایک مکمل \عددیء{2\pi} برقی زاویہ کے چکر کا وقت \عددیء{T} حاصل کر سکتے ہیں یعنی
\begin{align*}
t&=\frac{\theta}{\omega}\\
T&=\frac{2\pi}{2\pi f}=\frac{1}{f}
\end{align*}
اگر برقی رو کی تعدد \عددیء{50} ہو تو یہ مقناطیسی دباؤ کی موج ہر \عددیء{\tfrac{1}{50}=0.02} سیکنڈ میں ایک مکمل برقی چکر کاٹتی ہے یعنی یہ ایک سیکنڈ میں \عددیء{50} برقی چکر کاٹتی ہے۔

اس مثال میں برقی زاویہ کی بات ہوتی رہی۔ دو قطب کی آلوں میں برقی زاویہ \عددیء{\theta_e}  اور میکانی زاویہ \عددیء{\theta_m} برابر ہوتے ہیں۔ لہٰذا اگر دو قطب کی آلوں کی بات کی جائے تو مساوات  کے تحت ایک سیکنڈ میں مقناطیسی دباؤ کی موج \عددیء{f} برقی یا میکانی چکر کاٹے گی جہاں \عددیء{f} برقی رو کی تعدد ہے اور اگر \عددیء{P} قطب رکھنے والی آلوں کی بات کی جائے تو چونکہ
\begin{align}
\theta_e=\frac{P}{2} \theta_m
\end{align}
لہٰذا ایسے آلوں میں یہ مقناطیسی دباؤ کی موج ایک سیکنڈ میں \عددیء{f} مقناطیسی چکر یعنی \عددیء{\tfrac{2}{P}f} میکانی شکر کاٹے گی۔

اگر ہم برقی رو کی تعدد کو \عددیء{f_e} سے ظاہر کریں، مقناطیسی دباؤ کی موج کی چوٹی کے برقی زاویہ کو  \عددیء{\theta_e} اور اس کے میکانی زاویہ کو \عددیء{\theta_m} سے ظاہر کریں اور اسی طرح اسی مقناطیسی دباؤ کی موج کے گھومنے کی رفتار کو \عددیء{\omega_e} یا \عددیء{\omega_m} سے ظاہر کریں تو
\begin{gather}
\begin{aligned}
\omega_m&=\frac{2}{P} \omega_e \quad \si{\radian / \second}\\
f_m&=\frac{2}{P} f_e \quad \si{\hertz}\\
n&=\frac{120 f_e}{P} \quad rpm
\end{aligned}
\end{gather}
\عددیء{\omega_e} اس موج کی معاصر رفتار  برقی زاویہ فی سیکنڈ میں ہے جبکہ  \عددیء{\omega_m} یہی معاصر رفتار میکانی زاویہ فی سیکنڈ میں ہے۔اسی طرح \عددیء{f_e} اس موج کی برقی  معاصر رفتار برقی ہرٹز میں اور \عددیء{f_m} اس کی میکانی معاصر رفتار\فرہنگ{معاصر رفتار}\حاشیہب{synchronous speed}\فرہنگ{synchronous speed} میکانی ہرٹز میں ہے۔برقی معاصر رفتار \عددیء{f_e}  ہرٹز ہونے کا مطلب یہ ہے کہ ایک سیکنڈ میں یہ موج \عددیء{f_e} برقی چکر کا فاصلہ طے کرے گی جہاں ایک برقی چکر دو قطب کا فاصلہ یعنی \عددیء{2\pi}  ریڈیئن کا زاویہ ہے۔اسی طرح میکانی معاصر رفتار \عددیء{f_m} ہرٹز ہونے کا مطلب ہے کہ یہ موج ایک سیکنڈ میں \عددیء{f_m} میکانی چکر کا فاصلہ طے کرے گی۔ایک میکانی چکر عام زندگی میں ایک چکر کو ہی کہتے ہیں۔ اس مساوات میں \عددیء{n} میکانی چکر فی منٹ\فرہنگ{rpm}\حاشیہب{rpm, rounds per minute}  کو ظاہر کرتے ہیں۔یہ مساوات معاصر رفتار کی مساوات ہے۔

یہاں اس بات کا ذکر کرنا ضروری ہے کہ ہم \عددیء{q} دور کی لپٹی مشین جس کے لچھے \عددیء{\tfrac{2\pi}{q}} برقی زاویہ پر رکھے گئے ہوں اور جن میں \عددیء{q} دور کی برقی رو  ہو، ایک ہی سمت میں گھومتی مقناطیسی دباؤ کی موج کو جنم دیتی ہے جیسے ہم نے تین دور کی مشین کے لئے دیکھا۔ مزید یہ کہ اس موج کا حیطہ کسی ایک لچھے سے پیدا مقناطیسی دباؤ کے حیطہ  کے \عددیء{\tfrac{q}{2}}  گنا ہو گا اور اس کے گھومنے کی رفتار \عددیء{\omega_e=2\pi f} برقی ریڈیئن فی سیکنڈ ہو گی۔

\جزوحصہ{تین دور کی لپٹی مشین کا ترسیمی تجزیہ}
شکل  میں تین دور کی لپٹی مشین دکھائی گئی ہے۔ اس میں مثبت برقی رو کی سمتیں بھی دکھائی گئی ہیں، مثلاً \عددیء{a} شگاف میں برقی رو صفحہ سے عمودی سمت میں باہر جانب کو ہے اور یہ بات نقطہ سے واضح کی گئی ہے۔ اسی طرح \عددیء{-a} شگاف میں برقی دباؤ صفحہ سے عمودی سمت میں اندر کی جانب کو ہے اور یہ بات صلیب کے نشان سے واضح کی گئی ہے۔ اگر برقی رو مثبت ہو تو اس کی یہی سمت ہو گی اور اس سے پیدا مقناطیسی دباؤ \عددیء{\سمتیہ{\tau}_a} صفر زاویہ کی جانب ہو گا جیسے شکل میں دکھایا گیا ہے۔ لچھے میں برقی رو سے پیدا مقناطیسی دباؤ کی سمت دائیں ہاتھ کے قانون سے معلوم کی جا سکتی ہے۔ اب اگر اسی لچھے میں برقی رو منفی ہو تو اس کا مطلب ہے کہ برقی رو اُلٹ سمت میں ہے۔ یعنی اب برقی رو \عددیء{a} شگاف میں صفحہ کے عمودی سمت میں اندر کی جانب ہے اور \عددیء{-a} شگاف میں یہ صفحہ کے عمودی سمت میں باہر کی جانب کو ہے۔ لہٰذا اس برقی رو سے پیدا مقناطیسی دباؤ بھی پہلے سے اُلٹ سمت میں ہو گی یعنی یہ شکل میں دیئے گئے  \عددیء{\سمتیہ{\tau}_a}  کے بالکل اُلٹ سمت میں ہو گی۔ اس تذکرہ کا بنیادی مقصد یہ تھا کہ آپ پر یہ بات واضح ہو جائے کہ برقی رو کے منفی ہونے سے اس سے پیدا مقناطیسی دباؤ کی سمت اُلٹ ہو جاتی ہے۔

اس شکل میں لچھوں میں برقی رو اور مقناطیسی دباؤ یہ ہیں
\begin{gather}
\begin{aligned}
i_a&=I_0 \cos \omega t\\
i_b&=I_0 \cos (\omega t-120\degree)\\
i_c&=I_0 \cos (\omega t+120\degree)
\end{aligned}
\end{gather}
%
\begin{gather}
\begin{aligned}
\tau_a&=k_w \frac{4}{\pi}\frac{N i_a}{2}=k_w \frac{4}{\pi}\frac{N I_0}{2} \cos \omega t=\tau_0 \cos \omega t\\
\tau_b&=k_w \frac{4}{\pi}\frac{N i_b}{2}=k_w \frac{4}{\pi}\frac{N I_0}{2} \cos (\omega t-120\degree)=\tau_0 \cos (\omega t-120\degree)\\
\tau_c&=k_w \frac{4}{\pi}\frac{N i_c}{2}=k_w \frac{4}{\pi}\frac{N I_0}{2} \cos (\omega t+120\degree)=\tau_0 \cos (\omega t+120\degree)
\end{aligned}
\end{gather}
جبکہ ان کے مثبت سمتیں شکل میں دیئے گئے ہیں۔ اب ہم مختلف اوقات پر ان مقداروں کا حساب لگاتے ہیں اور ان کا کُل مجموعی مقناطیسی دباؤ حل کرتے ہیں۔

لمحہ \عددیء{t=0} پر ان مساوات سے ملتا ہے۔
\begin{gather}
\begin{aligned}
i_a&=I_0 \cos 0=I_0\\
i_b&=I_0 \cos (0-120\degree)=-0.5 I_0\\
i_c&=I_0 \cos (0+120\degree)=-0.5 I_0
\end{aligned}
\end{gather}
%
\begin{gather}
\begin{aligned}
\tau_a&=\tau_0 \cos 0=\tau_0\\
\tau_b&=\tau_0 \cos (0-120\degree)=-0.5 \tau_0\\
\tau_c&=\tau_0 \cos (0+120\degree)=-0.5 \tau_0
\end{aligned}
\end{gather}
یہاں رکھ کر زرا غور کریں۔اس لمحہ پر  \عددیء{i_a} مثبت ہے جبکہ \عددیء{i_b} اور \عددیء{i_c} منفی ہیں۔ لہٰذا \عددیء{i_a}  اُسی سمت میں ہے جو شکل  میں دکھایا گیا ہے جبکہ  \عددیء{i_b} اور \عددیء{i_c} شکل میں دیئے گئے سمتوں کے اُلٹ ہیں۔ ان تینوں برقی رو کی اس لمحہ پر درست سمتیں شکل  میں دکھائی گئی ہیں۔اس شکل میں تینوں مقناطیسی دباؤ بھی دکھائے گئے ہیں۔

کل مقناطیسی دباؤ با آسانی بذریعہ گراف، جمع سمتیات  سے معلوم کیا جا سکتا ہے یا پھر الجبرا کے ذریعہ ایسا کیا جا سکتا ہے۔
\begin{gather}
\begin{aligned}
\kvec{\tau}_a&=\tau_0 \ax\\
\kvec{\tau}_b&=0.5 \tau_0 \left[\cos (60\degree) \ax-\sin (60\degree)\ay \right]\\
\kvec{\tau}_c&=0.5 \tau_0 \left[\cos (60\degree) \ax+\sin (60\degree)\ay \right]
\end{aligned}
\end{gather}
%
\begin{align}
\kvec{\tau}=\kvec{\tau}_a+\kvec{\tau}_b+\kvec{\tau}_c=\frac{3}{2} \tau_0 \ax
\end{align}
کل مقناطیسی دباؤ ایک لچھے کے مقناطیسی دباؤ کے ڈیڑھ گنا ہے اور یہ صفر زاویہ پر ہے۔ اب ہم گھڑی کو چلنے دیتے ہیں اور کچھ لمحے بعد \عددیء{t_1} پر دوبارہ یہی سب حساب لگاتے ہیں۔ چونکہ مساوات  اور  میں متغیرہ \عددیء{t} کے بجائے \عددیء{\omega t} کا استعمال زیادہ آسان ہے لہٰذا ہم لمحہ \عددیء{t_1} کو یوں چنتے ہیں کہ \عددیء{\omega t_1=30\degree}  کے برابر ہو۔ ایسا کرنے سے ہمیں یہ دو مساواتوں سے حاصل ہوتا ہے۔
\begin{gather}
\begin{aligned}
i_a&=I_0 \cos 30\degree=\frac{\sqrt{3}}{2} I_0\\
i_b&=I_0 \cos (30\degree-120\degree)=0\\
i_c&=I_0 \cos (30\degree+120\degree)=-\frac{\sqrt{3}}{2} I_0
\end{aligned}
\end{gather}
%
\begin{gather}
\begin{aligned}
\tau_a&=\tau_0 \cos 30\degree=\frac{\sqrt{3}}{2} \tau_0\\
\tau_b&=\tau_0 \cos (30\degree-120\degree)=0\\
\tau_c&=\tau_0 \cos (30\degree+120\degree)=-\frac{\sqrt{3}}{2} \tau_0
\end{aligned}
\end{gather}
یہ شکل  میں دکھایا گیا ہے۔کل مقناطیسی دباؤ کا طول \عددیء{\tau} کو تکون کے ذریعہ یوں حل کیا جا سکتا ہے۔ اسی طرح اس کو زاویہ بھی اسی سے حاصل ہوتا ہے۔ یعنی
\begin{align}
\tau=\sqrt{\tau_a^2+\tau_c^2-2 \tau_a \tau_c \cos 120\degree}=\frac{3}{2}\tau_0
\end{align}
اور چونکہ اس تکون کے دو اطراف برابر ہیں لہٰذا اس کے باقی دو زاویہ بھی برابر اور \عددیء{30\degree} ہیں۔

ہم دیکھتے ہیں کہ کُل مقناطیسی دباؤ جو پہلے صفر زاویہ پر تھا اب وہ \عددیء{30\degree}  کے زاویہ پر ہے یعنی وہ گھڑی کے اُلٹ سمت گھوم گیا ہے۔ اگر ہم اسی طرح \عددیء{\omega t =40\degree} پر دیکھیں تو ہمیں کُل مقناطیسی دباؤ اب بھی \عددیء{\tfrac{3}{2}\tau_0} ہی ملے گا البتہ اب یہ \عددیء{45\degree} کے زاویہ پر ہو گا۔اگر کسی لمحہ جب \عددیء{\omega t=\theta\degree} کے برابر ہو یہ سارا حساب کیا جائے تو کُل مقناطیسی دباؤ اب بھی \عددیء{\tfrac{3}{2}\tau_0} ہی ملے گا البتہ یہ \عددیء{\theta\degree} کے زاویہ پر ہوگا۔

\حصہ{محرک برقی دباؤ}
یہاں محرک برقی دباؤ\حاشیہد{ابتدا میں حرکت سے پیدا ہونے والی برقی دباؤ کو محرک برقی دباؤ کہتے تھے۔اب روایتی طور پر کسی بھی طرح پیدہ کردہ برقی دباؤ کو محرک برقی دباؤ کہتے ہیں۔} کو ایک اور زاویہ سے پیش کیا جاتا ہے۔

\جزوحصہ{بدلتی رو والا برقی جنریٹر}
شکل  میں ایک بنیادی بدلتی رو جنریٹر\فرہنگ{جنریٹر!بدلتی رو}\حاشیہب{ac generator}\فرہنگ{generator!ac} دکھایا گیا ہے۔اس کا گھومتا برقی مقناطیس، خلائی درز میں سائن نما مقناطیسی دباؤ  پیدا کرتا ہے جس سے  درز میں سائن نما کثافتِ مقناطیسی بہاؤ \عددیء{B}  پیدا ہوتی ہے، یعنی
\begin{align}
B=B_0 \cos \theta_p
\end{align}
یہ مقناطیس \عددیء{\omega} زاویاتی رفتار سے گھوم رہا ہے۔یوں اگر ابتدائی لمحہ  \عددیء{t=0} پر  یہ \عددیء{a} لچھے کی سمت یعنی نقطہ دار اُفقی لکیر کی سمت میں ہو تو لمحہ  \عددیء{t} پر یہ گھوم کر زاویہ \عددیء{\theta_m=\omega t} پر ہوگا۔اس طرح مساوات   یوں بھی لکھا جا سکتا ہے۔
\begin{gather}
\begin{aligned}
B&=B_0 \cos (\theta-\theta_m)\\
&=B-0 \cos (\theta -\omega t)
\end{aligned}
\end{gather}
شکل  میں \عددیء{B} کو زاویہ \عددیء{\theta} اور \عددیء{\theta_p}  کے ساتھ گراف کیا گیا ہے۔ اسی گراف میں لچھا \عددیء{a} بھی دکھایا گیا ہے۔اس شکل میں نقطہ دار لکیر لمحہ \عددیء{t=0} پر \عددیء{B} دکھا رہا ہے جب گھومتے برقی مقناطیس کا محور اور اس لچھے کا محور ایک ہی سمت میں ہوں جبکہ ٹھوس لکیر اسی \عددیء{B} کو کسی  بھی لمحہ \عددیء{t} پر دکھا رہا ہے اور اس لمحہ پر برقی مقناطیس کے محور اور لچھے کے محور کے مابین \عددیء{\vartheta} درجے کا زاویہ ہے۔ یہ زاویہ برقی مقناطیس کے گھومنے کی رفتار \عددیء{\omega} پر منحصر ہے یعنی
\begin{align}
\vartheta=\omega t
\end{align}
لمحہ \عددیء{t=0} پر لچھے میں سے زیادہ سے زیادہ مقناطیسی بہاؤ گزر رہی ہے۔ اگر خلائی درز بہت باریک ہو، تو اس کے اندر اور باہر جانب کے رداس تقریباً یکساں ہوں گے۔ برقی مقناطیس کے محور سے اس خلائی درز تک کا اوسط رداسی فاصلہ اگر \عددیء{\rho} ہو اور برقی مقناطیس کا دُھرے\فرہنگ{دُھرا}\حاشیہب{axle}\فرہنگ{axle} کی سمت میں محوری لمبائی\فرہنگ{محور}\حاشیہب{axial length} \عددیء{l} ہو تو اس لچھے میں وہی مقناطیسی بہاؤ ہو گا جو اس خلائی درز میں  \عددیء{-\tfrac{\pi}{2} < \theta < \tfrac{\pi}{2}}  کے مابین ہے۔ لمحہ \عددیء{t=0} پر اسے یوں معلوم کیا جا سکتا ہے۔
\begin{gather}
\begin{aligned}
\phi_a(0)&=\int_{-\frac{\pi}{2}}^{+\frac{\pi}{2}} \kvec{B} \cdot \dif \kvec{S}\\
&=\int_{-\frac{\pi}{2}}^{+\frac{\pi}{2}} (B_0 \cos \theta_p)(l \rho \dif \theta_p)\\
&=B_0 l \rho \left. \sin \theta_p \right|_{-\frac{\pi}{2}}^{+\frac{\pi}{2}}\\
&=2 B_0 l \rho\\
&=\phi_0
\end{aligned}
\end{gather}
 یہی حساب اگر لمحہ \عددیء{t} پر کی جائے تو کچھ یوں ہو گا۔
\begin{gather}
\begin{aligned}
\phi_a(t)&=\int_{-\frac{\pi}{2}-\vartheta}^{+\frac{\pi}{2}-\vartheta} \kvec{B} \cdot \dif \kvec{S}\\
&=\int_{-\frac{\pi}{2}-\vartheta}^{+\frac{\pi}{2}-\vartheta} (B_0 \cos \theta_p)(l \rho \dif \theta_p)\\
&=B_0 l \rho \left. \sin \theta_p \right|_{-\frac{\pi}{2}-\vartheta}^{+\frac{\pi}{2}-\vartheta}\\
&=2 B_0 l \rho \cos \vartheta\\
&=2 B_0 l \rho \cos \omega t
\end{aligned}
\end{gather}
جہاں \عددیء{\vartheta=\omega t} لیا گیا ہے۔اسی مساوات کو یوں بھی حل کیا جا سکتا ہے
\begin{gather}
\begin{aligned}
\phi_a(t)&=\int_{-\frac{\pi}{2}}^{+\frac{\pi}{2}} \kvec{B} \cdot \dif \kvec{S}\\
&=\int_{-\frac{\pi}{2}}^{+\frac{\pi}{2}} (B_0 \cos (\theta-\omega t))(l \rho \dif \theta)\\
&=B_0 l \rho \left. \sin (\theta-\omega t) \right|_{-\frac{\pi}{2}}^{+\frac{\pi}{2}}\\
&=B_0 l \rho \left[\sin \left(\frac{\pi}{2}-\omega t \right )-\sin \left (-\frac{\pi}{2}-\omega t \right) \right]\\
&=2 B_0 l \rho \cos \omega t
\end{aligned}
\end{gather}
اس مرتبہ تکمل زاویہ \عددیء{\theta} کے ساتھ کیا گیا ہے۔ انہیں مساوات  کی مدد سے یوں لکھا جا سکتا ہے۔
\begin{align}
\phi_a(t)=2 B_0 l \rho \cos \omega t=\phi_0 \cos \omega t
\end{align}
بالکل مساوات  کی طرح ہم  \عددیء{b} اور \عددیء{c} لچھوں کے لئے  بھی مقناطیسی بہاؤ کی مساواتیں حل کر سکتے ہیں۔شکل  میں \عددیء{a} لچھے میں زاویہ \عددیء{-\tfrac{\pi}{2}} سے \عددیء{+\tfrac{\pi}{2}}  تک کا مقناطیسی بہاؤ گزرتا ہے۔ اس لئے \عددیء{\phi_a(t)}  معلوم کرنے کے لئے مساوات  میں تکمل کے حدود یہی رکھے گئے تھے۔ اسی شکل سے واضح ہے کہ \عددیء{b} لچھے کے تکمل کے حدود 
\عددیء{+\tfrac{\pi}{6}}  اور \عددیء{+\tfrac{7 \pi}{6}} جبکہ \عددیء{c} کے حدود \عددیء{+\tfrac{5\pi}{6}} اور \عددیء{+\tfrac{11\pi}{6}} ہیں۔یہ زاویے ریڈیئن میں دیئے گئے ہیں۔یوں
\begin{gather}
\begin{aligned}
\phi_b(t)&=\int_{\frac{\pi}{6}}^{\frac{7\pi}{6}} \kvec{B} \cdot \dif \kvec{S}\\
&=\int_{\frac{\pi}{6}}^{\frac{7\pi}{6}} (B_0 \cos (\theta-\omega t))(l \rho \dif \theta)\\
&=B_0 l \rho \left. \sin (\theta-\omega t) \right|_{\frac{\pi}{6}}^{\frac{7\pi}{6}}\\
&=B_0 l \rho \left[\sin \left(\frac{7\pi}{6}-\omega t \right )-\sin \left (\frac{\pi}{6}-\omega t \right) \right]\\
&=2 B_0 l \rho \cos (\omega t-\frac{2\pi}{3})
\end{aligned}
\end{gather}
اور
\begin{gather}
\begin{aligned}
\phi_c(t)&=\int_{\frac{5\pi}{6}}^{\frac{11\pi}{6}} \kvec{B} \cdot \dif \kvec{S}\\
&=\int_{\frac{5\pi}{6}}^{\frac{11\pi}{6}} (B_0 \cos (\theta-\omega t))(l \rho \dif \theta)\\
&=B_0 l \rho \left. \sin (\theta-\omega t) \right|_{\frac{5\pi}{6}}^{\frac{11\pi}{6}}\\
&=B_0 l \rho \left[\sin \left(\frac{11\pi}{6}-\omega t \right )-\sin \left (\frac{5\pi}{6}-\omega t \right) \right]\\
&=2 B_0 l \rho \cos (\omega t+\frac{2\pi}{3})
\end{aligned}
\end{gather}
اگر ایک لچھے کے \عددیء{N} چکر ہوں تو اس میں پیدا برقی دباؤ کو یوں معلوم کیا جا سکتا ہے۔
\begin{gather}
\begin{aligned}
\lambda_a&=N \phi_a (t)=N \phi_0 \cos \omega t\\
\lambda_b&=N \phi_b (t)=N \phi_0 \cos (\omega t-120\degree)\\
\lambda_c&=N \phi_c (t)=N \phi_0 \cos (\omega t+120\degree)
\end{aligned}
\end{gather}
ان مساوات میں \عددیء{\tfrac{2\pi}{3}} ریڈیئن کو \عددیء{120\degree} لکھا گیا ہے۔ان سے لچھوں میں پیدا امالی برقی دباؤ کا حساب یوں لگایا جا سکتا ہے۔
\begin{gather}
\begin{aligned}
e_a(t)&=-\frac{\dif \lambda_a}{\dif t}=\omega N \phi_0 \sin \omega t\\
e_b(t)&=-\frac{\dif \lambda_b}{\dif t}=\omega N \phi_0 \sin (\omega t-120\degree)\\
e_c(t)&=-\frac{\dif \lambda_c}{\dif t}=\omega N \phi_0 \sin (\omega t+120\degree)
\end{aligned}
\end{gather}
ان مساوات کو یوں بھی لکھ سکتے ہیں
\begin{gather}
\begin{aligned}
e_a(t)&=\omega N \phi_0 \cos (\omega t -90\degree)\\
e_b(t)&=\omega N \phi_0 \cos (\omega t+150\degree)\\
e_c(t)&=\omega N \phi_0 \cos (\omega t+30\degree)
\end{aligned}
\end{gather}
یہ مساوات تین دوری محرک برقی دباؤ  کو ظاہر کرتے ہیں جو آپس میں \عددیء{120\degree} زاویہ پر ہیں۔ان سب کا حیطہ \عددیء{E_0} یکساں ہے جہاں
\begin{align}
E_0=\omega N \phi_0
\end{align}
اور ان برقی دباؤ کی موثر قیمت\فرہنگ{موثر قیمت}\حاشیہب{rms}\فرہنگ{rms}
\begin{align}
E_{rms}=\frac{E_0}{\sqrt{2}}=\frac{2\pi f N \phi_0}{\sqrt{2}}=4.44 f N \phi_0
\end{align}
ہو گی۔ مساوات  سائن نما برقی دباؤ کو ظاہر کرتا ہے۔ چونکہ \عددیء{\phi=B A}ہوتا ہے  لہٰذا یہ مساوات بالکل مساوات  کی طرح ہے۔ اگرچہ مساوات  یہ سوچ کر حاصل کیا گیا کہ خلائی درز میں مقناطیسی بہاؤ صرف برقی مقناطیس کی وجہ سے ہے تاہم برقی دباؤ کا اس سے کوئی تعلق نہیں کہ خلائی درز میں مقناطیسی بہاؤ کس طرح وجود میں آئی اور یہ مساوات ان حالات کے لئے بھی درست ہے جہاں یہ مقناطیسی بہاؤ جنریٹر کے ساکن حصے میں پیدا ہوئی ہو یا ساکن اور حرکت پذیر دونوں حصوں میں پیدا ہوئی ہو۔

مساوات  ہمیں ایک گچھ لچھے میں پیدا برقی دباؤ دیتی ہے۔ اگر لچھا تقسیم شدہ ہو تو اس کے مختلف شگافوں میں موجود اس لچھے کے حصوں میں برقی دباؤ ہم مرحلہ نہیں ہوں گے لہٰذا ان سب کا مجموعی برقی دباؤ ان سب کا حاصل جمع نہیں ہو گا بلکہ اس سے قدرِ کم ہو گا۔ اس مساوات کو ہم ایک پھیلے لچھے کے لئے یوں لکھ سکتے ہیں۔
\begin{align}
E_{rms}=4.44 k_w f N \phi_0
\end{align}
تین دور برقی جنریٹروں کے \عددیء{k_w} کی قیمت \عددیء{0.85} تا \عددیء{0.95} ہوتی ہے۔ یہ مساوات ہمیں ایک دور کی برقی دباؤ دیتی ہے۔ تین دور برقی جنریٹروں میں ایسے تین لچھوں کے جوڑے ہوتے ہیں اور ان کو \عددیء{Y} یعنی ستارا نما یا \عددیء{\Delta} یعنی تکونی جوڑا جاتا ہے۔

\جزوحصہ{یک سمتی رو برقی جنریٹر}
ہر گھومنے والا برقی جنریٹر بنیادی طور پر بدلتی رو جنریٹر ہی ہوتا ہے۔ البتہ جہاں یک سمتی برقی دباؤ\فرہنگ{برقی دباؤ!یک سمتی}\حاشیہب{DC voltage}\فرہنگ{voltage!DC}  کی ضرورت ہو وہاں مختلف طریقوں سے بدلتی برقی دباؤ  کو یک سمتی برقی دباؤ میں تبدیل کیا جاتا ہے۔ ایسا الیکٹرانکس کے ذریعہ جنریٹر کے باہر برقیاتی سمت کار\فرہنگ{سمت کار!برقیاتی}\حاشیہب{rectifier}\فرہنگ{rectifier} کی مدد سے  کیا جا سکتا ہے یا پھر میکانی طریقے سے میکانی سمت کار\فرہنگ{سمت کار!میکانی}\حاشیہب{commutator}\فرہنگ{commutator}  کی مدد سے جنریٹر کے اندر ہی کیا جا سکتا ہے۔ مساوات  میں دیئے گئے برقی دباؤ کو یک سمتی برقی دباؤ میں تبدیل کیا جائے تو یہ شکل  کی طرح ہو گا۔

\ابتدا{مثال}
شکل  میں یک سمتی برقی دباؤ دکھائی گئی ہے۔اس یک سمتی برقی دباؤ کی اوسط حاصل کریں۔

حل:
\begin{align*}
E_{\textup{اوسط}}=\frac{1}{\pi} \int_{0}^{\pi} \omega N \phi_0 \sin \omega t \dif (\omega t)=\frac{2 \omega N \phi_0}{\pi}
\end{align*}
\انتہا{مثال}
یک سمتی برقی جنریٹر پر باقائدہ تبصرہ کتاب کے باب  میں کیا جائے گا۔

\حصہ{ہموار قطب مشینوں میں مروڑ}
اس حصے میں ہم ایک کامل مشین میں مروڑ\فرہنگ{مروڑ}\حاشیہب{torque}\فرہنگ{torque} کا حساب لگائیں گے۔ ایسا دو طریقوں سے کیا جا سکتا ہے۔ ہم مشین کو دو مقناطیس سمجھ کر ان کے مابین قوتِ کشش، قوتِ دفع اور مروڑ کا حساب لگا سکتے ہیں یا پھر اس میں ساکن اور گھومتے لچھوں کو امالہ سمجھ کر باب چار کی طرح توانائی اور کو توانائی کے استعمال سے اس کا حساب لگائیں۔ پہلے توانائی کا طریقہ استعمال کرتے ہیں۔

\جزوحصہ{توانائی کے طریقے سے  میکانی مروڑ کا حساب}
یہاں ہم ایک دور کی مشین کی بات کریں گے۔ اس سے حاصل جوابات کو با آسانی زیادہ دور کی آلوں پر لاگو کیا جا سکتا ہے۔ شکل میں ایک دور کی کامل مشین دکھائی گئی ہے۔ کسی بھی لمحہ اس کی دو لچھوں میں کچھ زاویہ ہو گا جسے \عددیء{\theta} سے ظاہر کیا گیا ہے۔ خلائی درز ہر جگہ یکساں ہے لہٰذا یہاں اُبھرے قطب کے اثرات کو نظر انداز کیا جائے گا۔ مزید یہ کہ مرکز  کی \عددیء{\mu_r \to \infty}  تصور کی گئی ہے لہٰذا لچھوں کی امالہ صرف خلائی درز کی مقناطیسی مستقل\فرہنگ{مقناطیسی مستقل}\حاشیہب{magnetic constant, permeability} \عددیء{\mu_0} پر منحصر ہے۔

اس طرح ساکن لچھے کی امالہ \عددیء{L_{aa}} اور گھومے لچھے کی امالہ \عددیء{L_{rr}} مقررہ ہیں جبکہ ان کا مشترکہ امالہ \عددیء{L_{ar}(\theta)} زاویہ \عددیء{\theta} پر منحصر ہو گا۔ جب \عددیء{\theta=0} یا \عددیء{\theta=\mp 2\pi}  کے برابر ہو تو ایک لچھے کا سارہ مقناطیسی بہاؤ دوسرے لچھے سے بھی گزرتا ہے۔ ایسے حالت میں ان کا مشترکہ امالہ زیادہ سے زیادہ ہو گا جسے \عددیء{L_{ar0}} لکھتے ہیں۔ جب  \عددیء{\theta=\mp 180\degree} ہو اس لمحہ ایک بار پھر ایک لچھے کا سارہ مقناطیسی بہاؤ دوسرے لچھے سے بھی گزرتا ہے البتہ اس لمحہ اس کی سمت اُلٹ ہوتی ہے لہٰذا اب ان کا مشترکہ امالہ بھی منفی ہو گا یعنی \عددیء{-L_{ar0}} اور جب \عددیء{\theta=\mp 90\degree} ہو  تب ان کا مشترکہ امالہ صفر ہو گا۔ اگر ہم یہ ذہن میں رکھیں کہ خلائی درز میں  مقناطیسی بہاؤ سائن نما ہے تب
\begin{align}
L_{ar}=L_{ar0} \cos \theta
\end{align}
ہو گا۔ہم ساکن اور گھومتے لچھوں کی اِرتَباطِ بہاؤ کو یوں لکھ سکتے ہیں
\begin{gather}
\begin{aligned}
\lambda_a&=L_{aa} i_a+L_{ar}(\theta) i_r=L_{aa} i_a+L_{ar0} \cos (\theta) i_r\\
\lambda_r&=L_{ar}(\theta) i_a+L_{rr} i_r=L_{ar0} \cos (\theta) i_a+L_{rr} i_r
\end{aligned}
\end{gather}
اگر ساکن لچھے کی مزاحمت \عددیء{R_a} اور گھومتے لچھے کی مزاحمت \عددیء{R_r} ہو تو ہم ان لچھوں کے سروں پر دیئے گئے برقی دباؤ کو یوں لکھ سکتے ہیں۔
\begin{gather}
\begin{aligned}
v_a&=i_a R_a +\frac{\dif \lambda_a}{\dif t}=i_a R_a+L_{aa} \frac{\dif i_a}{\dif t}+L_{ar0} \cos \theta \frac{\dif i_r}{\dif t}-L_{ar0}  i_r \sin \theta  \frac{\dif \theta}{\dif t}\\
v_r&=i_r R_r +\frac{\dif \lambda_r}{\dif t}=i_r R_r+L_{ar0} \cos \theta \frac{\dif i_a}{\dif t}-L_{ar0} i_a \sin \theta  \frac{\dif \theta}{\dif t}+L_{rr} \frac{\dif i_r}{\dif t}
\end{aligned}
\end{gather}
یہاں \عددیء{\theta} برقی زاویہ ہے اور وقت کے ساتھ اس کی تبدیلی رفتار \عددیء{\omega} کو ظاہر کرتی ہے یعنی
\begin{align}
\frac{\dif \theta}{\dif t}=\omega
\end{align}
میکانی مروڑ بذریعہ کو توانائی حاصل کی جا سکتی ہے۔ کو توانائی مساوات  سے حاصل ہوتی ہے۔ یہ مساوات موجودہ استعمال کے لئے یوں لکھا جا سکتا ہے۔
\begin{align}
W_m'=\frac{1}{2} L_{aa} i_a^2+\frac{1}{2} L_{rr} i_r^2+L_{ar0} i_a i_r \cos \theta
\end{align}
اس سے میکانی مروڑ \عددیء{T_m} یوں حاصل ہوتا ہے۔
\begin{align}
T_m=\frac{\partial W_m'(\theta_m,i_a,i_r)}{\partial \theta_m}=\frac{\partial W_m'(\theta,i_a,i_r)}{\partial \theta} \frac{\partial \theta}{\partial \theta_m}
\end{align}
چونکہ \عددیء{P} قطب مشینوں کے لئے
\begin{align}
\theta=\frac{P}{2} \theta_m
\end{align}
لہٰذا ہمیں مساوات  سے ملتا ہے
\begin{align}
T_m=-\frac{P}{2} L_{ar0} i_a i_r \sin \left(\frac{P}{2} \theta_m\right)
\end{align}
اس مساوات میں مروڑ \عددیء{T_m} منفی ہے۔ اس کا مطلب ہے کہ اگر کسی لمحہ پر ساکن اور گھومتے لچھوں کے مقناطیسی بہاؤ کے درمیان زاویہ مثبت ہو تو ان کے مابین مروڑ منفی ہو گا یعنی مروڑ ان دونوں مقناطیسی بہاؤ کو ایک سمت میں رکھنے کی کوشش کرے گا۔

\جزوحصہ{مقناطیسی بہاؤ سے میکانی مروڑ کا حساب}
شکل  میں دو قطب والی ایک دور کی مشین دکھائی گئی ہے۔ اس شکل میں بائیں جانب صرف گھومتے لچھے میں برقی رو ہے۔ اس لچھے کا مقناطیسی بہاؤ تیر کے نشان سے دکھایا گیا ہے، یعنی تیر اس مقناطیس کے محور کو ظاہر کرتا ہے۔ یہاں اگر صرف گھومتے حصے پر توجہ دی جائے تو یہ واضح ہے کہ گھومتا حصہ ایک مقناطیس کی مانند ہے جس کے شمالی اور جنوبی قطبین شکل میں دیئے گئے ہیں۔ اسی طرح شکل میں دائیں جانب صرف ساکن لچھے میں برقی رو ہے۔ اگر اس مرتبہ صرف ساکن حصے پر توجہ دی جائے تو اس کے بائیں جانب سے مقناطیسی بہاؤ نکل کر خلائی درز میں داخل ہوتی ہے، لہٰذا یہی اس کا شمالی قطب ہے اور اس مقناطیس کا محور بھی اسی تیر کی سمت میں ہے۔

یہاں یہ واضح رہے کہ اگرچہ گچھ لچھے دکھائے گئے ہیں  لیکن درحقیقت دونوں لچھوں کے مقناطیسی دباؤ سائن-نما ہی ہیں اور تیر کے نشان ان مقناطیسی دباؤ کی موج کے چوٹی کو ظاہر کرتے ہیں۔

شکل  میں اب دونوں لچھوں میں برقی رو ہے۔ یہ واضح ہے کہ یہ بالکل دو مقناطیسوں کی طرح ہے اور ان کے اُلٹ قطبین کے مابین قوتِ کشش ہو گا، یعنی یہ دونوں لچھے ایک ہی سمت میں ہونے کی کوشش کریں گے۔

یہاں یہ زیادہ واضح ہے کہ یہ دو مقناطیس کوشش کریں گے کہ \عددیء{\theta_{ar}} صفر کے برابر ہو یعنی ان کا میکانی مروڑ \عددیء{\theta_{ar}} کے اُلٹ سمت میں ہو گا۔ یہی کچھ مساوات  کہتا تھا۔

ان برقی مقناطیسوں کے مقناطیسی دباؤ کو اگر ان کے مقناطیسی محور کی سمت میں \عددیء{\سمتیہ{\tau}_a} اور \عددیء{\سمتیہ{\tau}_r} سے ظاہر کیا جائے جہاں \عددیء{\tau_a} اور \عددیء{\tau_r} مقناطیسی دباؤ کے چوٹی کے برابر ہوں تو خلاء میں کُل مقناطیسی دباؤ \عددیء{\سمتیہ{\tau}_{ar}} ان  کا جمع سمتیات ہو گا جیسے شکل میں دکھایا گیا ہے۔ اس کا طول \عددیء{\tau_{ar}} کوسائن کے قلیہ\حاشیہب{cosine law}  سے یوں حاصل ہوتا ہے۔
\begin{gather}
\begin{aligned}
\tau_{ar}^2&=\tau_a^2+\tau_r^2-2\tau_a \tau_r \cos (180\degree-\theta_{ar})\\
&=\tau_a^2+\tau_r^2+2\tau_a \tau_r \cos \theta_{ar}
\end{aligned}
\end{gather}
خلائی درز میں یہ کُل مقناطیسی دباؤ، مقناطیسی شدت \عددیء{H_{ar}} کو جنم دے گا جو اس قلیہ سے حاصل ہوتا ہے۔
\begin{align}
\tau_{ar}=H_{ar} l_g
\end{align}
\عددیء{H_{ar}} مقناطیسی شدت کی چوٹی کو ظاہر کرتا ہے۔ اب جہاں خلاء میں مقناطیسی شدت \عددیء{H} ہو وہاں مقناطیسی کو-توانائی کی کثافت \عددیء{\tfrac{\mu_0}{2}H^2} ہوتی ہے۔ خلائی درز میں اوسط کو-توانائی کی کثافت اس خلائی درز میں \عددیء{H^2} کی اوسط ضربِ \عددیء{\tfrac{\mu_0}{2}} ہو گی۔ کسی بھی سائن نما موج \عددیء{H=H_0 \cos \theta} کے \عددیء{H^2} کا اوسط \عددیء{H_{\textup{اوسط}}^2} یوں حاصل کیا جاتا ہے۔
\begin{gather}
\begin{aligned}
H_{\textup{اوسط}}^2&=\frac{1}{\pi} \int_{-\frac{\pi}{2}}^{+\frac{\pi}{2}} H^2 \dif \theta\\
&=\frac{1}{\pi} \int_{-\frac{\pi}{2}}^{+\frac{\pi}{2}} H_0^2 \cos^2 \theta \dif \theta\\
&=\frac{H_0^2}{\pi} \int_{-\frac{\pi}{2}}^{+\frac{\pi}{2}} \frac{1+\cos 2 \theta}{2} \dif \theta\\
&=\frac{H_0^2}{\pi}  \left.  \frac{\theta +\frac{\sin 2 \theta}{2}}{2} \right|_{-\frac{\pi}{2}}^{+\frac{\pi}{2}}\\
&=\frac{H_0^2}{2}
\end{aligned}
\end{gather}
لہٰذا خلائی درز میں اوسط کو-توانائی کی کثافت \عددیء{\tfrac{\mu_0}{2}\tfrac{H_{ar}^2}{2}} ہو گی اور اس خلاء میں کُل کو-توانائی اس اوسط کو-توانائی ضربِ خلاء کی حجم کے برابر ہو گا یعنی
\begin{align}
W_m'=\frac{\mu_0}{2} \frac{H_{ar}^2}{2} 2 \pi r l_g l=\frac{\mu_0 \pi r l}{2l_g} \tau_{ar}^2
\end{align}
اس مساوات میں خلائی درز کی رداسی لمبائی \عددیء{l_g} ہے اور اس کی دُھرے\حاشیہب{axis} کی سمت میں محوری لمبائی\حاشیہب{axial length} \عددیء{l} ہے۔ محور سے خلاء کی اوسط رداسی فاصلہ \عددیء{r} ہے۔ مزید یہ کہ \عددیء{r\gg l_g} ہے۔ اس طرح خلاء میں رداسی سمت میں کثافتِ مقناطیسی بہاؤ کی تبدیلی کو نذر انداز کیا جا سکتا ہے۔ اس مساوات کو ہم مساوات  کی مدد سے یوں لکھ سکتے ہیں۔
\begin{align}
W_m'=\frac{\mu_0 \pi r l}{2 l_g} \left(\tau_a^2+\tau_r^2+2\tau_a \tau_r \cos \theta_{ar} \right) 
\end{align}
اس سے میکانی مروڑ یوں حاصل کیا جا سکتا ہے
\begin{align}
T_m=\frac{\partial W_m'}{\partial \theta_{ar}}=-\frac{\mu_0 \pi r l}{l_g} \tau_a \tau_r \sin \theta_{ar}
\end{align}
یہ حساب دو قطب والی مشین کے لئے لگایا گیا ہے۔\عددیء{P} قطب والے مشین کے لئے یہ مساوات ہر جوڑی قطب کا میکانی مروڑ دیتا ہے لہٰذا ایسے مشین کے لئے ہم لکھ سکتے ہیں
\begin{align}
T_m=-\frac{P}{2}\frac{\mu_0 \pi r l}{l_g} \tau_a \tau_r \sin \theta_{ar}
\end{align}
یہ ایک بہت اہم مساوات ہے۔ اس کے مطابق مشین کا میکانی مروڑ اس کے ساکن اور گھومتے لچھوں کے مقناطیسی دباؤ کے چوٹی کے براہ راست متناسب ہے۔ اسی طرح یہ ان دونوں کے درمیان برقی زاویہ \عددیء{\theta_{ar}} کے سائن کے بھی براہ راست متناسب ہے۔ منفی میکانی مروڑ کا مطلب ہے کہ یہ زاویہ  \عددیء{\theta_{ar}} کے الٹ جانب ہے یعنی یہ میکانی مروڑ اس زاویہ کو کم کرنے کی جانب کو ہے۔ مشین کے ساکن اور گھومتے حصوں پر ایک برابر مگر الٹ سمتوں میں میکانی مروڑ ہوتا ہے البتہ ساکن حصے کا مروڑ مشین کے وجود کے ذریعہ زمین تک منتقل ہو جاتا ہے جبکہ گھومتے حصے کا میکانی مروڑ اس حصے کو گھماتا ہے۔

چونکہ مقناطیسی دباؤ برقی رو کے براہ راست متناسب ہے لہٰذا  \عددیء{\tau_a} اور \عددیء{i_a} آپس میں براہ راست متناسب ہیں جبکہ \عددیء{\tau_r} اور \عددیء{i_r} آپس میں براہ راست متناسب ہیں۔ اس سے یہ ظاہر ہوتا ہے کہ مساوات  اور  ایک جیسے ہیں۔ درحقیقت یہ ثابت کیا جا سکتا ہے کہ یہ دونوں بالکل برابر ہیں۔

شکل  میں ایک بار پھر ساکن اور گھومتے لچھوں کے مقناطیسی دباؤ دکھائے گئے ہیں۔ شکل میں بائیں جانب تکون \عددیء{\Delta AEC} اور \عددیء{\Delta BEC} میں \عددیء{CE} مشترکہ ہے اور ان دو تکونوں سے واضح ہے کہ
\begin{align}
CE=\tau_r \sin \theta_{ar}=\tau_{ar} \sin \theta_a
\end{align}
اس مساوات کی مدد سے مساوات  یوں لکھا جا سکتا ہے۔
\begin{align}
T_m=-\frac{P}{2}\frac{\mu_0 \pi r l}{l_g} \tau_a \tau_{ar}  \sin \theta_a
\end{align}
اسی طرح شکل  کے دائیں جانب تکون \عددیء{\Delta MWQ} اور تکون \عددیء{\Delta SWQ} میں \عددیء{WQ} کا طرف مشترکہ ہے اور ان دو تکونوں سے واضح ہے کہ
\begin{align}
WQ=\tau_a \sin \theta_{ar}=\tau_{ar} \sin \theta_r
\end{align}
اب اس مساوات کی مدد سے مساوات  یوں لکھا جا سکتا ہے۔
\begin{align}
T_m=-\frac{P}{2}\frac{\mu_0 \pi r l}{l_g} \tau_r \tau_{ar}  \sin \theta_r
\end{align}
مساوات  مساوات  اور مساوات  کو ایک جگہ لکھتے ہیں۔
\begin{gather}
\begin{aligned}
T_m&=-\frac{P}{2}\frac{\mu_0 \pi r l}{l_g} \tau_a \tau_r \sin \theta_{ar}\\
T_m&=-\frac{P}{2}\frac{\mu_0 \pi r l}{l_g} \tau_a \tau_{ar}  \sin \theta_a\\
T_m&=-\frac{P}{2}\frac{\mu_0 \pi r l}{l_g} \tau_r \tau_{ar}  \sin \theta_r
\end{aligned}
\end{gather}
ان مساوات سے یہ واضح ہے کہ میکانی مروڑ کو دونوں لچھوں کے مقناطیسی دباؤ اور ان کے مابین زاویہ کی شکل میں لکھا جا سکتا ہے یا پھر ایک لچھے کی مقناطیسی دباؤ اور کُل مقناطیسی دباؤ اور ان دو کے مابین زاویہ کی شکل میں لکھا جا سکتا ہے۔

اس بات کو یوں بیان کیا جاسکتا ہے کہ میکانی مروڑ دو مقناطیسی دباؤ کے آپس میں ردعمل کی وجہ سے وجود میں آتا ہے اور یہ ان مقناطیسی دباؤ کی چوٹی اور ان کے مابین زاویہ پر منحصر ہوتا ہے۔

مقناطیسی دباؤ، مقناطیسی شدت، کثافت مقناطیسی بہاؤ اور مقناطیسی بہاؤ سب کا آپس میں تعلق رکھتے ہیں لہٰذا ان مساوات کو کئی مختلف طریقوں سے لکھا جا سکتا ہے۔ مثلاً خلائی درز میں کُل مقناطیسی دباؤ \عددیء{\tau_{ar}} اور  وہاں کثافت مقناطیسی بہاؤ \عددیء{B_{ar}} کا تعلق
\begin{align}
B_{ar}=\frac{\mu_0 \tau_{ar}}{l_g}
\end{align}
استعمال کر کے مساوات  کے آخری جز کو یوں لکھا جا سکتا ہے
\begin{align}
T_m=-\frac{P}{2} \pi r l \tau_r B_{ar} \sin \theta_r
\end{align}
مقناطیسی آلوں میں مقناطیسی مرکز کی مقناطیسی مستقل \عددیء{\mu}  کی محدود صلاحیت کی وجہ سے مرکز میں کثافت مقناطیسی بہاؤ تقریباً ایک ٹسلہ تک ہی بڑھائی جا سکتی ہے۔ لہٰذا مشین بناتے وقت اس حد کو مد نظر رکھنا پڑتا ہے۔ اسی طرح گھومتے لچھے کا مقناطیسی دباؤ اس لچھے میں برقی رو پر منحصر ہوتا ہے۔ اس برقی رو سے لچھے کی مزاحمت میں برقی توانائی ضائع ہوتی ہے جس سے یہ لچھا گرم ہوتا ہے۔ برقی رو کو اس حد تک بڑھایا جا سکتا ہے جہاں تک اس لچھے کو ٹھنڈا کرنا ممکن ہو۔ لہٰذا مقناطیسی دباؤ کو اس حد کے اندر رکھنا پڑتا ہے۔ چونکہ اس مساوات میں یہ دو بہت ضروری حدیں واضح طور پر سامنے ہیں اس لئے یہ مساوات مشین بنانے کی غرض سے بہت اہم ہے۔

اس مساوات کی ایک اور بہت اہم شکل اب دیکھتے ہیں۔ ایک قطب پر مقناطیسی بہاؤ \عددیء{\phi_P}  ایک قطب پر اوسط کثافت مقناطیسی بہاؤ \عددیء{B_{\textup{اوسط}}} ضربِ ایک قطب کا رقبہ  \عددیء{A_P} ہوتا ہے۔ جہاں
\begin{align}
B_{\textup{اوسط}}&=\frac{1}{\pi} \int_{-\frac{\pi}{2}}^{+\frac{\pi}{2}} B_0 \cos \theta \dif \theta=\frac{2 B_0}{\pi}\\
A_P&=\frac{2\pi rl}{P}
\end{align}
لہٰذا
\begin{align}
\phi_P=\frac{2 B_0}{\pi}\frac{2\pi rl}{P}
\end{align}
اور
\begin{align}
T_m=-\frac{\pi}{2} \left(\frac{P}{2} \right)^2 \phi_{ar} \tau_r \sin \theta_r
\end{align}
یہ مساوات معاصر مشینوں  کے لئے بہت کار آمد ہے۔
